\documentclass[11pt,a4paper]{article}
\usepackage[margin=1in]{geometry}
\usepackage[utf8]{inputenc}
\usepackage{amsmath}
\usepackage[english]{babel}
\usepackage{amsthm}
\usepackage{amsfonts}
\usepackage{algorithm}
\usepackage{algpseudocode}
\usepackage{amssymb}
\usepackage{enumerate}
\usepackage[hidelinks]{hyperref}
\usepackage{cancel}

\hypersetup{
    colorlinks=true,
    linkcolor=blue,
    filecolor=magenta,      
    urlcolor=blue,
    pdftitle={Differential Geometry},
    pdfpagemode=FullScreen,
}

% \graphicspath{ {./images/} }

\author{Jayadev Naram}
\title{Differential Geometry}

\begin{document}

\date{}
\maketitle
\tableofcontents

\newpage

\theoremstyle{plain}
\newtheorem{theorem}{Theorem}[section]
\newtheorem{corollary}{Corollary}[theorem]

\newtheorem{lemma}[theorem]{Lemma}
\newtheorem{proposition}[theorem]{Proposition}
\newtheorem{assume}{Assumption}

\theoremstyle{definition}
\newtheorem{definition}[theorem]{Definition}
\newtheorem{example}[theorem]{Example}
\newtheorem{remark}[theorem]{Remark}

\newcommand{\R}{\mathbb{R}}
\newcommand{\B}{\mathbb{B}}
\newcommand{\A}{\mathcal{A}}
\newcommand{\M}{\mathcal{M}}
\newcommand{\N}{\mathcal{N}}
\newcommand{\h}{\mathcal{H}}
\newcommand{\T}{\mathcal{T}}
\newcommand{\Prob}{\mathbb{P}}
\newcommand{\Dist}{\mathcal{D}}
\newcommand{\perpProj}{\mathcal{P}^\perp}
\newcommand{\bb}{\mathbb{B}}
\newcommand{\Sprod}{\mathbb{S}_{xy}}
\newcommand{\highlight}[1]{\underline{\textit{\textbf{#1}}}}
\newcommand{\mapping}[3]{#1:#2\rightarrow #3}
\newcommand{\doubt}{\highlight{[??]}}
% \newcommand{\bigvert}[2]{#1{\raisebox{-.5ex}{$|$}_{#2}}}
\newcommand{\Cinf}{\mathcal{C}^\infty}
\newcommand{\bigvert}[2]{\left.#1\right|_{#2}}
\newcommand{\sdnn}[1]{${#1}$}
\newcommand{\bsdnn}[1]{$\boldsymbol{#1}$}
\newcommand{\ifthen}[2]{\textbf{(#1)}\boldsymbol{\implies}\textbf{(#2)}}
\newcommand{\bsdn}[1]{\boldsymbol{#1}}
\newcommand{\forward}{$(\implies)$\ }
\newcommand{\converse}{$(\impliedby)$\ }
\newcommand{\Lt}[1]{\underset{#1\rightarrow 0}{Lt}}
\newcommand{\norm}[1]{\|#1\|}
\newcommand{\dparder}[2]{\dfrac{\partial #1}{\partial x_{#2}}}
\newcommand{\fparder}[2]{\frac{\partial #1}{\partial x_{#2}}}
\newcommand{\parder}[2]{\partial #1/\partial x_{#2}}
\newcommand{\parop}[1]{\dfrac{\partial}{\partial x_{#1}}}
\newcommand{\innerproduct}[2]{\langle #1, #2 \rangle}
\newcommand{\metric}[2]{[#1, #2]}
\newcommand{\genst}{St_B(n,p)}
\newcommand{\igenst}[1]{St_{B_{#1}}(n_{#1},p)}
\newcommand{\realmat}[2]{\R^{#1\times #2}}
\newcommand{\Skew}{\mathcal{S}_{skew}(p)}
\newcommand{\Sym}{\mathcal{S}_{sym}(p)}
\newcommand{\XperpB}{X_{B^\perp}}
\newcommand{\polarRetr}{R^{polar}_X}
\newcommand{\qrRetr}{R^{QR}_X}
\newcommand{\vectransport}{\mathcal{T}}
\newcommand{\grad}{\text{grad}\,}
\newcommand{\hess}{\text{Hess}\,}

\section{Smooth Manifolds}

\begin{definition}
A topological space $\M$ is said to be \highlight{locally Euclidean} of dimension $n$ if every point of $\M$ has a neighborhood in $\M$ that is homeomorphic to an open subset of $\R^n$.
\end{definition}

\begin{lemma}
A topological space $\M$ is locally Euclidean of dimension $n$ if and only if either of the following properties holds:
\begin{enumerate}[(a)]
    \item Every point of $\M$ has a neighborhood homeomorphic to an open ball in $\R^n$.
    \item Every point of $\M$ has a neighborhood homeomorphic to $\R^n$.
\end{enumerate}
\end{lemma}

\begin{proof}
% (Sketch) For (a) one need to show that any open subset of $\R^n$ is homeomorphic to the unit open ball in $\R^n$. Note that the unit open ball is homeomorphic to any open ball in $\R^n$. For (b) it suffices to show that unit open ball in $\R^n$ is homeomorphic to $\R^n$.
(a) \forward Let $x\in \M$ and suppose that there a neighborhood of $x$ in $\M$ that is homeomorphic to an open subset $U$ in $\R^n$. 

\end{proof}

\begin{definition}
Suppose $\M$ is a topological space. We say $\M$ is a \highlight{topological manifold} of dimension $n$ or a \highlight{topological \bsdnn{n}-manifold} if it has the following properties:
\begin{enumerate}[(a)]
    \item $\M$ is a {Hausdorff space}.
    \item $\M$ is a {second-countable}.
    \item $\M$ is {locally Euclidean of dimension $n$}.
\end{enumerate}
A \highlight{coordinate chart} (or just a \highlight{chart}) on $\M$ is a pair $(U,\varphi)$, where $U$ is an open subset of $\M$ and $\mapping{\varphi}{U}{\hat{U}}$ is a homeomorphism from $U$ to an open subset $\hat{U} = \varphi(U)\subseteq \R^n$. The set $U$ is called a \highlight{coordinate domain} or a \highlight{coordinate neighborhood} of each of its points. The map $\varphi$ is called a \highlight{(local) coordinate map}, and the component functions $(x^1,\ldots,x^n)$ of $\varphi$, defined by $\varphi(p) = (x^1(p),\ldots,x^n(p))$, are called \highlight{local coordinates} on $U$.
\end{definition}

\begin{proposition}
A nonempty $n$-dimensional topological manifold cannot be homeomorphic to an $m$-dimensional manifold unless $m=n$.
\end{proposition}

\begin{example}
Here are some examples of topological manifolds.
\begin{enumerate}[(i)]
    \item Open subset of a topological n-manifold.
    % \item Euclidean Spaces.
    \item Graphs of Continuous Functions.
    \item Spheres.
    \item Projective Spaces.
    \item Product Manifolds.
\end{enumerate}
\end{example}

\begin{definition}
Let $\M$ be a topological $n$-manifold. If $(U,\varphi), (V,\psi)$ are two charts such that $U\cap V\neq \emptyset$, the composite map $\mapping{\psi\circ \varphi^{-1}}{\varphi(U\cap V)}{\psi(U\cap V)}$ is called \highlight{transition map from $\boldsymbol{\varphi}$ to $\boldsymbol{\psi}$}. Two charts $(U,\varphi),(V,\psi)$ are said to be \highlight{smoothly compatible} if either $U\cap V = \emptyset$ or the transition map $psi\circ\varphi^{-1}$ is a ($\Cinf$)diffeomorphism.
\end{definition}

\begin{remark}
In the above definition, since $\psi(U\cap V)$ and $\varphi(U\cap V)$ are open subsets of $\R^n$, smoothness of the transition map $\psi\circ\varphi^{-1}$ can be interpreted in the ordinary sense of having continuous partial derivatives of all orders.
\end{remark}

\begin{definition}
We define an \highlight{atlas for $\boldsymbol\M$} to be a collection of charts whose domains cover $\M$. An atlas $\A$ is called a \highlight{smooth atlas} if any two charts in $\A$ are smoothly compatible with each other.
\end{definition}

\begin{remark}
To show that an atlas is smooth, we need only verify that each transition map $\psi\circ\varphi^{-1}$ is smooth whenever $(U,\varphi), (V,\psi)$ are charts in $\A$ such that $U\cap V\neq \emptyset$; once we have proved this, it follows that $\psi\circ\varphi^{-1}$ is a diffeomorphism because its  inverse $(\psi\circ\varphi^{-1})^{-1} = \varphi\circ\psi^{-1}$ is one of the transition maps we have already shown to be smooth. Alternatively, given two particular charts $(U,\varphi), (V,\psi)$, it is often easiest to show that they are smoothly compatible by verifying that $\psi\circ\varphi^{-1}$ is smooth and injective with nonsingular Jacobian at each point, and appealing to a variant inverse function theorem(\cite[Corollary C.36]{JohnLee}).
\end{remark}

\begin{definition}
Let $\M$ be a topological manifold. A smooth atlas $\A$ on $\M$ is \highlight{maximal} if it is not properly contained in any larger smooth atlas.
\end{definition}

\begin{remark}
If $\A$ is a maximal smooth atlas on $\M$, then any chart that is smoothly compatible with every chart in $\A$ is already in $\A$.
\end{remark}

\begin{definition}
Let $\M$ be a topological manifold. A \highlight{smooth structure on $\boldsymbol\M$} is a maximal smooth atlas. A \highlight{smooth manifold} is a pair $(\M,\A)$ where $\M$ is a topological manifold and $\A$ is a smooth structure on $\M$. Any chart $(U,\varphi)$ in $\A$ is called a \highlight{smooth chart} and the corresponding coordinate map $\varphi$ and the domain $U$ of $\varphi$ are called \highlight{smooth coordinate map} and \highlight{smooth coordinate domain} or \highlight{smooth coordinate neighborhood} respectively.
\end{definition}

\begin{theorem}
Let $\M$ be a topological manifold.
\begin{enumerate}[(a)]
    \item Every smooth atlas $\A$ on $\M$ is contained in a unique maximal smooth atlas, called the \highlight{smooth structure determined by $\A$}.
    \item Two smooth atlases for $\M$ determine the same smooth structure iff their union is a smooth atlas.
\end{enumerate}
\end{theorem}

\begin{example}
Here are some examples of smooth manifolds.
\begin{enumerate}[(i)]
    \item Euclidean Spaces.
    \item Finite-Dimensional Vector Spaces.
    \item Space of Matrices.
    \item Open Submanifolds.
    \item The General Linear Group.
    \item Matrices of Full Rank.
    \item Spaces of Linear Maps.
    \item Graphs of Continuous Functions.
    \item Spheres.
    \item Level Sets.
    \item Projective Spaces.
    \item Smooth Product Manifolds.
    \item Grassmann Manifolds.
\end{enumerate}
\end{example}

Solve the exercise questions 1-1 to 1-10 from \cite[Ch 1]{JohnLee}.

\section{Smooth Maps}

\begin{remark}
For the sake of convenience, we reserve the word \highlight{function} for a map whose codomain is $\R$ (a \highlight{real-valued function}) or $\R^k$ for some $k>1$ (a \highlight{vector-valued function}). Either of the words \highlight{map} or \highlight{mapping} can mean any type of map, such as a map between arbitrary manifolds.
\end{remark}

\begin{definition}\label{def:smooth_function}
Suppose $\M$ is a smooth $n$-manifold, $k$ is a nonnegative integer, and $\mapping{f}{\M}{\R^k}$ is any function. We say that $f$ is a \highlight{smooth function} if for every $p\in\M$, there exists a smooth chart $(U,\varphi)$ for $\M$ whose domain contains $p$ and such that the composite function $f\circ\varphi^{-1}$ is smooth on the open subset $\hat{U} = \varphi(U)\subseteq\R^n$.
\end{definition}

\begin{remark}
The most important special case is that of smooth real-valued functions $\mapping{f}{\M}{\R}$; the set of all such functions is denoted by $\Cinf(\M)$. Because sums and constant multiples of smooth functions are smooth, $\Cinf(\M)$ is a vector space over $\R$.
\end{remark}

\begin{proposition}\label{prop:smooth_coordinate_rep}
Let $\M$ be a smooth manifold, and suppose $\mapping{f}{\M}{\R^k}$ is a smooth function. Then $\mapping{f\circ\varphi^{-1}}{\varphi(U)}{\R^k}$ is smooth for every smooth chart $(U,\varphi)$ for $\M$.
\end{proposition}

% \begin{proof}
% The function $\mapping{f}{\M}{\R^k}$ is a smooth function, then for all $p\in\M$ there exists a smooth chart $(U,\varphi)$ for $\M$ such that $p\in U$ and $\mapping{f\circ \varphi^{-1}}{\varphi(U)}{\R^k}$ is smooth, where $\varphi(U)$ is an open subset of $\R^n$. Consider any other smooth chart on
% \end{proof}

\begin{definition}
Given a function $\mapping{f}{\M}{\R^k}$, and a chart $(U,\varphi)$ for $\M$, the function $\mapping{\hat{f}}{\varphi(U)}{\R^k}$ defined by $\hat{f}(x) = f\circ\varphi^{-1}(x)$ is called the \highlight{coordinate representation of $\boldsymbol f$}.
\end{definition}

\begin{remark}
By Def.\ref{def:smooth_function}, $f$ is smooth iff its coordinate representation is smooth in some smooth chart around each point. By Prop.\ref{prop:smooth_coordinate_rep}, smooth functions have smooth coordinate representations in every smooth chart.
\end{remark}

\begin{proposition}
Let $U$ be an open submanifold of $\R^n$ with its standard smooth manifold structure. Then a function $\mapping{f}{U}{\R^k}$ is smooth in the sense of Def.\ref{def:smooth_function} iff it is smooth in the sense of ordinary calculus.
\end{proposition}

\begin{definition}\label{def:smooth_map}
Let $\M, \N$ be smooth manifolds, and let $\mapping{F}{\M}{\N}$ be any map. We say that $F$ is a \highlight{smooth map} if for every $p\in\M$, there exist smooth charts $(U,\varphi)$ containing $p$ and $(V,\psi)$ containing $F(p)$ such that $F(U)\subseteq V$ and the composite map $\psi\circ F\circ \varphi^{-1}$ is smooth from $\varphi(U)$ to $\psi(V)$.
\end{definition}

\begin{remark}
Def.\ref{def:smooth_function} can be viewed as a special case of Def.\ref{def:smooth_map} by taking $\N=V=\R^k$ and $\psi=\mapping{Id}{\R^k}{\R^k}$.
\end{remark}

\begin{proposition}
Every smooth map is continuous.
\end{proposition}

\begin{proposition}[\highlight{Equivalent Characterizations of Smoothness}]
Suppose $\M$ and $\N$ are smooth manifolds, and $\mapping{F}{\M}{\N}$ is a map. Then $F$ is smooth iff either of the following conditions is satisfied:
\begin{enumerate}[(a)]
    \item For every $p\in\M$, there exists smooth charts $(U,\varphi)$ containing $p$ and $(V,\psi)$ containing $F(p)$ such that $U\cap F^{-1}(V)$ is open in $\M$ and the composite map $\psi\circ F\circ \varphi^{-1}$ is smooth from $\varphi(U\cap F^{-1}(V))$ to $\psi(V)$.
    \item $F$ is continuous and there exist smooth atlases $\{(U_\alpha,\varphi_\alpha)\}$ and $\{(V_\beta,\psi_\beta)\}$ for $\M$ and $\N$, respectively, such that for each $\alpha$ and $\beta$, $\psi_\beta\circ F\circ \varphi_\alpha^{-1}$ is smooth from $\varphi_\alpha(U_\alpha\cap F^{-1}(V_\beta))$ to $\psi_\beta(V_\beta)$.
\end{enumerate}
\end{proposition}

\begin{proposition}[\highlight{Smoothness is Local}]
Let $\M,\N$ be smooth manifolds, and let $\mapping{F}{\M}{\N}$ be a map.
\begin{enumerate}[(a)]
    \item If every point $p\in\M$ has a neighborhood $U$ such that the restriction $F|_U$ is smooth, then $F$ is smooth.
    \item Conversely, if $F$ is smooth, then its restriction to every open subset is smooth.
\end{enumerate}
\end{proposition}

\begin{proposition}[\highlight{Gluing Lemma for Smooth Maps}]
Let $\M,\N$ be smooth manifolds, and let $\{U_\alpha\}_{\alpha\in A}$ be an open cover of $\M$. Suppose that for each $\alpha\in A$, we are given a smooth map $\mapping{F_\alpha}{U_\alpha}{\N}$ such that the maps agree on overlaps: $F_\alpha|_{U_\alpha\cap U_\beta} = F_\beta|_{U_\alpha\cap U_\beta}$ for all $\alpha$ and $\beta$. Then there exists a unique smooth map $\mapping{F}{\M}{\N}$ such that $F|_{U_\alpha} = F_\alpha$ for each $\alpha\in A$.
\end{proposition}

\begin{definition}
Given a map $\mapping{F}{\M}{\N}$, and smooth charts $(U,\varphi)$ and $(V,\psi)$ for $\M$ and $\N$, respectively, the function $\mapping{\hat{F}}{\varphi(U\cap F^{-1}(V))}{\psi(V)}$ defined by $\hat{F}(x) = \psi\circ F\circ\varphi^{-1}(x)$ is called the \highlight{coordinate representation of $\boldsymbol F$}.
\end{definition}

\begin{proposition}
Suppose $\mapping{F}{\M}{\N}$ is a smooth map between smooth manifolds. Then the coordinate representation of $F$ with respect to every pair of smooth charts for $\M$ and $\N$ is smooth.
\end{proposition}

\begin{proposition}
Let $\M$, $\N$, and $\mathcal{P}$ be smooth manifolds.
\begin{enumerate}[(a)]
    \item Every constant map $\mapping{c}{\M}{\N}$ is smooth.
    \item The identity map of $\M$ is smooth.
    \item If $U \subseteq \M$ is an open submanifold, then the inclusion map $U\hookrightarrow \M$ is smooth.
    \item If $\mapping{F}{\M}{\N}$ and $\mapping{G}{\N}{\mathcal{P}}$ are smooth, then so is $\mapping{G\circ F}{\M}{\mathcal{P}}$.
\end{enumerate}
\end{proposition}

\begin{proposition}
Suppose $\M_1,\ldots, \M_k$ and $\N$ are smooth manifolds. For each $i$, let $\mapping{\pi_i}{\M_1\times\ldots\times\M_k}{\M_i}$ denote the projection onto the $\M_i$ factor. A map $\mapping{F}{\N}{\M_1\times\ldots\times\M_k}$ is smooth iff each of the component maps $F_i = \mapping{\pi_i\circ F}{\N}{\M_i}$ is smooth.
\end{proposition}

\section{Partitions of Unity}

\section{Tangent Vectors}

\begin{definition}
Given a point $x\in \R^n$, the \highlight{geometric tangent space} to $\R^n$ at $x$, denoted by $\R^n_x$, is the set 
$$\R^n_x = \{x\}\times \R^n = \{(x,v):\;v\in\R^n\}.$$ 
A \highlight{geometric tangent vector} in $\R^n$ is an element of $\R^n_x$ for some $x \in \R^n$. As a matter of notation, we abbreviate $(x,v)$ as $v_x$ or $\bigvert{v}{x}$. We think of $v_x$ as the vector $v$ with its initial point at $x$.
\end{definition}

\begin{remark}
The set $\R^n_x$ is a real vector space under the natural operations
$$v_x + w_x = (v+w)_x,\;\;\; c(v_x) = (cv)_x.$$
Consequently, the vectors $\bigvert{e_i}{x}, i = 1,\ldots, n$, are a basis for $\R^n_x$.
\end{remark}

\begin{definition}
If $x$ is a point of $\R^n$, a map $\mapping{w}{\Cinf(\R^n)}{\R}$ is called a \highlight{derivation at $\boldsymbol{x}$} if it is linear over $\R$ and satisfies the following product rule:
$$
w(fg) = f(x)wg + g(x)wf.
$$
Let $T_x\R^n$ denote the set of all derivation of $\Cinf(\R^n)$ at $x$.
\end{definition}

\begin{remark}
Clearly, $T_x\R^n$ is a vector space under the operations
$$ (w_1+w_2)f = w_1f+w_2f, \;\;\; (cw)f = c(wf). $$
\end{remark}

\begin{remark}\label{note:directional_derivative}
For any geometric tangent vector $v_x\in \R^n_x$ we define a derivation to be a map which takes the directional derivative of any $f\in\Cinf(\R^n)$ in the direction $v$ at $x$:
$$
\bigvert{D_v}{x} f = Df(x)[v] = \bigvert{\dfrac{d}{dt}}{t=0} f(x+tv) = 
\lim_{t\rightarrow 0} \dfrac{f(x+tv)-f(x)}{t}.
$$
It is indeed true that it is linear over $\R$ since for any $f, g\in \Cinf(\R^n)$ and $\alpha,\beta\in\R$, we have 
\begin{align*}
\bigvert{D_v}{x}(\alpha f+\beta g) = D(\alpha f+\beta g)(x)[v] \
=& \lim_{t\rightarrow 0} \dfrac{\alpha f(x+tv)+\beta g(x+tv) - \alpha f(x)-\beta g(x)}{t} \\
=& \alpha \lim_{t\rightarrow 0} \dfrac{f(x+tv)- f(x)}{t} + \beta\lim_{t\rightarrow 0} \dfrac{g(x+tv) - g(x)}{t} \\
=& \alpha Df(x)[v]+\beta Dg(x)[v] 
= \alpha \bigvert{D_v}{x} f+ \beta\bigvert{D_v}{x} g.
\end{align*}
One can also note that this map satisfies the product rule(or chain rule):
$$ \bigvert{D_v}{x}(fg) = f(x)\bigvert{D_v}{x}g + g(x)\bigvert{D_v}{x}f. $$
If $v_a = \sum_{i=1}^n v^{(i)} \bigvert{e_i}{a}$ in terms of the standard basis, then by the chain rule $\bigvert{D_v}{a}f$ can be written more concretely as
$$
\bigvert{D_v}{a}f = \sum_{i=1}^n v^{(i)} \dfrac{\partial f}{\partial x^{(i)}}(a).
$$
\end{remark}

\begin{lemma}[\highlight{Properties of Derivations}]
Suppose $x\in\R^n,w\in T_x\R^n$, and $f,g\in\Cinf(\R^n)$.
\begin{enumerate}[(a)]
    \item If $f$ is a constant function, then $wf = 0$.
    \item If $f(x) = g(x) = 0$, then $w(fg) = 0$.
\end{enumerate}
\end{lemma}

\begin{proposition}
Let $x\in \R^n$.
\begin{enumerate}[(a)]
    \item For each geometric tangent vector $v_x\in\R^n_x$, the map $\mapping{\bigvert{D_v}{x}}{\Cinf(R^n)}{\R}$ defined in Note \ref{note:directional_derivative} is a derivation at $x$.
    \item The map $v_x\mapsto \bigvert{D_v}{x}$ is an isomorphism from $\R^n_x$ onto $T_x\R^n$.
\end{enumerate}
\end{proposition}

\begin{corollary}
For any $a\in \R^n$, the n derivations
$$
\bigvert{\dfrac{\partial}{\partial x^{(1)}}}{a},\ldots, \bigvert{\dfrac{\partial}{\partial x^{(n)}}}{a} \text{ defined by } \bigvert{\dfrac{\partial}{\partial x^{(i)}}}{a} f = \dfrac{\partial f}{\partial x^{(i)}}(a)
$$
form a basis for $T_a\R^n$, which therefore has dimension $n$.
\end{corollary}

\begin{definition}
Let $\M$ be a smooth manifold, and let $p$ be a point of $\M$. A linear map $\mapping{v}{\Cinf(\M)}{\R}$ is called a \highlight{derivation at $\boldsymbol{p}$} if it satisfies
\begin{equation*}
    v(fg) = f(p)vg + g(p) vf, \text{ for all } f,g\in \Cinf(\M).
\end{equation*}
The set of all derivations of $\Cinf(\M)$ at $p$, denoted by $T_p\M$, is a vector space called the \highlight{tangent space to $\boldsymbol{\M}$ at $\boldsymbol{p}$}. An element of $T_p\M$ is called a \highlight{tangent vector at $\boldsymbol{p}$}.
\end{definition}

\begin{lemma}[\highlight{Properties of Tangent Vectors on Manifolds}]
Suppose $\M$ is a smooth manifold, $p \in\M$, $v \in T_p\M$, and $f, g \in\Cinf(\M)$.
\begin{enumerate}[(a)]
    \item If $f$ is a constant function, then $vf = 0$.
    \item If $f(p) = g(p) = 0$, then $v(fg) = 0$.
\end{enumerate}
\end{lemma}

\begin{definition}
If $\M$ and $\N$ are smooth manifolds and $\mapping{F}{\M}{\N}$ is a smooth map, for each $p\in\M$ we define a map $\mapping{dF_p}{T_p\M}{T_{F(p)}\N}$, called the \highlight{differential of \bsdnn{F} at \bsdnn{p}}, as follows. Given $v \in T_p\M$, we let $dF_p(v)$ be the derivation at $F(p)$ that acts on $f\in\Cinf(\N)$ by the rule
\begin{equation*}
    dF_p(v)(f) = v(f \circ F).
\end{equation*}
\end{definition}

\begin{remark}
The operator $\mapping{dF_p}{\Cinf(\N)}{\R}$ is linear because $v$ is, and is a derivation at $F(p)$ because for any $f, g\in \Cinf(\N)$ we have
\begin{align*}
dF_p(v)(fg) 
&= v((fg)\circ F) = v((f\circ F)(g\circ F)) \\
&= f(F(p))v(g\circ F) + g(F(p)) v(f\circ F) \\
&= f(F(p))dF_p(v)(g) + g(F(p)) dF_p(v)(f).
\end{align*}
\end{remark}

\begin{proposition}[\highlight{Properties of Differentials}]
Let $\M, \N,$ and $\mathcal{P}$ are smooth manifolds, let $\mapping{F}{\M}{\N}$ and $\mapping{G}{\N}{\mathcal{P}}$ be smooth maps, and let $p\in\M$,
\begin{enumerate}
    \item $\mapping{dF_p}{T_p\M}{T_{F(p)}\N}$ is linear.
    \item $d(G\circ F)_p = \mapping{dG_{F(p)}\circ dF_p}{T_p\M}{T_{G\circ F(p)}\mathcal{P}}$.
    \item $d(Id_\M) = \mapping{Id_{T_p\M}}{T_p\M}{T_p\M}$.
    \item If $F$ is a diffeomorphism, then $\mapping{dF_p}{T_p\M}{T_{F(p)}\N}$ is an isomorphism, and $(dF_p)^{-1} = d(F^{-1})_{F(p)}$.
\end{enumerate}
\end{proposition}

\begin{proposition}[\highlight{Tangent vectors act locally}]
Let $\M$ be a smooth manifold, $p \in \M$ and $v\in T_p\M$. If $f, g \in \Cinf(M)$ agree on some neighborhood of $p$, then $vf = vg$.
\end{proposition}

\begin{proposition}[\highlight{The Tangent Space to an Open Submanifold}]
Let $\M$ be a smooth manifold, let $U\subseteq \M$ be an open subset, and let $\mapping{\iota}{U}{\M}$ be the inclusion map. For every $p \in U$, the differential $\mapping{d\iota_p}{T_pU}{T_p\M}$ is an isomorphism.
\end{proposition}

\begin{proposition}[\highlight{Dimension of the Tangent Space}]
If $\M$ is an $n$-dimensional smooth manifold, then for each $p \in \M$, the tangent space $T_p\M$ is an $n$-dimensional vector space.
\end{proposition}

\begin{proposition}[\highlight{The Tangent Space to a Vector Space}]
Suppose $V$ is a finite-dimensional vector space with its standard smooth structure. For any vector $v\in V$, we define a  map $\mapping{\bigvert{D_v}{a}}{\Cinf(V)}{\R}$ by 
\begin{equation*}
    \bigvert{D_v}{a} f = \bigvert{\dfrac{d}{dt}}{t=0} f(a+tv).
\end{equation*}
Then, the map $v\mapsto \bigvert{D_v}{a}$ defined above is an isomorphism from $v$ to $T_aV$, such that for any linear map $\mapping{L}{V}{W}$ we have $dL_a(\bigvert{D_v}{a}) = D_{\bigvert{Lv}{La}}$.
\end{proposition}

\begin{proposition}[\highlight{The Tangent Space to a Product Manifold}]
Let $\M_1,\ldots,\M_k$ be smooth manifolds, and for each $j$, let $\mapping{\pi_j}{\M_1\times\ldots\times\M_k}{\M_j}$ denote the projection onto the $\M_j$ factor. For any point $p = (p_1,\ldots,p_k)\in \M_1\times\ldots\times\M_k$, the map 
\begin{equation*}
    \mapping{\alpha}{T_p(\M_1\times\ldots\times\M_k)}{T_{p_1}\M_1\oplus\ldots\oplus T_{p_k}\M_k}
\end{equation*}
defined by 
\begin{equation*}
    \alpha(v) = (d(\pi_1)_p(v),\ldots,d(\pi_k)_p(v))
\end{equation*}
is an isomorphism.
\end{proposition}

\begin{proposition}
Let $\M$ be a smooth $n$-manifold, and let $p\in \M$. Then $T_p\M$ is an $n$-dimensional vector space, and for any smooth chart $(U,(x^{(i)}))$ containing $p$, the coordinate vectors $\bigvert{\partial/\partial x^{(1)}}{p},\ldots,\bigvert{\partial/\partial x^{(n)}}{p}$ form a basis for $T_p\M$.
\end{proposition}



\newpage

\begin{thebibliography}{9}
\bibitem{JohnLee}
John M. Lee, Introduction to Smooth Manifolds.

% \bibitem{Prop 1}
% [Prop 1(b)]Closure and continuity - \href{https://math.stackexchange.com/questions/114462/}{math.SE}

% \bibitem{Seq Lemma (b)}
% [Sequence Lemma (b)] \href{https://math.stackexchange.com/questions/1876224/}{math.SE}

\end{thebibliography}

\end{document}