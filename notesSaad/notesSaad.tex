\documentclass[10pt,a4paper]{article}
\usepackage[utf8]{inputenc}
\usepackage{amsmath}
\usepackage[english]{babel}
\usepackage{amsthm}
\usepackage{amsfonts}
\usepackage{algorithm}
\usepackage[noend]{algpseudocode}
\usepackage{amssymb}
\usepackage{enumerate}

\author{Jayadev Naram}
\title{Iterative Methods}

\begin{document}

\maketitle 

\maketitle
 
\tableofcontents

\newpage

%\part{}
  
\newtheorem{theorem}{Theorem}
\newtheorem{corollary}{Corollary}[theorem]
\newtheorem{lemma}[theorem]{Lemma}
\newtheorem{mydef}{Definition}
\newtheorem*{remark}{Remark}
\newtheorem{prop}{Proposition}

\section{General Projection Methods}

Let A $\in\mathbb{R}^{n\times n}$ and $\mathcal{K}\;and\;\mathcal{L}$ be two m-dimensional subspaces of $\mathbb{R}^n$. A projection technique onto the subspace $\mathcal{K}$ and orthogonal to $\mathcal{L}$ with an initial guess $x_0$ is a process which finds an approximate solution $\tilde{x}$ by imposing the conditions that $\tilde{x}$ belong to $x_0+\mathcal{K}$ and that the new residual vector be orthogonal to $\mathcal{L}$, i.e, $$find\;\;\;\tilde{x}\in x_0+\mathcal{K},\;such\;that\;\;b-A\tilde{x}\perp \mathcal{L}.$$
$$\tilde{x}=x_0+\delta,\;\delta\in\mathcal{K}$$
$$(r_0-A\delta,w)=0,\;\forall\,w\in\mathcal{L},\;where\;r_0=b-Ax_0.$$

Let $V=[v_1,\cdots,v_m]_{n\times m}\;and\;W=[w_1,\cdots,w_m]_{n\times m}$ whose column-vectors form a basis of $\mathcal{K}$ and $\mathcal{L}$, respectively.
Then approximate solution can be written as:
$$\tilde{x}=x_0+Vy,$$
where y can found from the orthogonality constraint:
$$W^TAVy=W^Tr_0.$$
If $W^TAV$ is non-singular, then $\tilde{x}=x_0+V(W^TAV)^{-1}W^Tr_0.$

\begin{algorithm}
\caption{Prototype Projection Method}
\begin{algorithmic}[1]
\Repeat
	\State Select a pair of subspaces $\mathcal{K}\;and\;\mathcal{L}$
	\State Choose basis $V$=$[v_1,\cdots,v_m],\;W$=$[w_1,\cdots,w_m]\;for\;\mathcal{K}\;and\;\mathcal{L}$
	\State $r\gets b-Ax$
	\State $y\gets (W^TAV)^{-1}W^Tr$
	\State $x\gets x+Vy$
\Until{Convergence}
\end{algorithmic}
\end{algorithm}

Non-singularity of A is not sufficient condition for non-singularity of $W^TAV$.

\begin{prop}
Let $A,\;\mathcal{L}\;and\;\mathcal{K}$ satisfy either one of the two following conditions:
\begin{enumerate}[i.]
\item A is SPD and $\mathcal{L}=\mathcal{K}$, or
\item A is non-singular and $\mathcal{L}=A\mathcal{K}$.
\end{enumerate}
Then $B=W^TAV$ is non-singular for any bases V and W of $\mathcal{K}\;and\;\mathcal{L}$.
\end{prop}

\begin{proof}
Consider case(i). Since $\mathcal{L}=\mathcal{K}$, then $W=VG$, where G is a non-singular $m\times m$ matrix. Then $B=W^TAV={G^T}V^TAV$. Since A is SPD, so is $V^TAV$ and since G is non-singular, B is non-singular. \\
Now, consider case(ii). Since $\mathcal{L}=A\mathcal{K}$, then $W=AVG$, where G is a non-singular $m\times m$ matrix. Then $B=W^TAV={G^T}(AV)^TAV$. Since A is non-singular, then $(AV)_{n\times m }$ full rank matrix and so is $(AV)^TAV$ and therefore, B is non-singular.
\end{proof}

\begin{theorem}
Assume that A is SPD and $\mathcal{L}=\mathcal{K}.$ Then a vector $\tilde{x}$ is the result of an (orthogonal) projection method onto $\mathcal{K}$ with the  starting vector $x_0$ iff it minimizes the A-norm if the error over $x_0+\mathcal{K},$ i.e, iff
$$\tilde{x} = \underset{x\in x_0+\mathcal{K}}{\arg\min}\|x_*-x\|_A=\underset{x\in x_0+\mathcal{K}}{\arg\min}(A(x_*-x),x_*-x)^{\frac{1}{2}}$$
\end{theorem}

\begin{proof}
First we prove that if $\tilde{x}$ minimizes A-norm of the error, then it is the result of orthogonal projection method with $x_0$ onto $\mathcal{K}$. Assume columns of V to be basis vectors of $\mathcal{K}$, then the objective function can be written as:
\begin{align*}
E(x) &= (A(x_*-x),x_*-x)^{\frac{1}{2}},\qquad(x\in x_0+\mathcal{K}) \\
\implies E(y) &= (A(x_*-x_0-Vy),x_*-x_0-Vy)^{\frac{1}{2}},\;\;(y\in\mathbb{R}^m) \\
\implies E^2(y) &= (A(x_*-x_0-Vy),x_*-x_0-Vy), \\
&= (x_*-x_0-Vy)^TA(x_*-x_0-Vy),\\
&= c + 2y^TV^T(Ax_0-Ax_*) + y^TV^TAVy, \\
&= c - 2y^TV^T(b-Ax_0) + y^TV^TAVy = f(y), \\
\frac{\partial f(y)}{\partial y} = 0 &\implies V^T(b-A(x_0+Vy)) = 0 \\
&\implies V^T(b-A\tilde{x}) = 0 \\
&\implies b-A\tilde{x}\perp \mathcal{K}.
\end{align*}
Therefore the residue of vector which minimizes A-norm of error over $x_0+\mathcal{K}$ is orthogonal to $\mathcal{K}$, therefore it is the result of orthogonal projection method onto $\mathcal{K}$ starting with $x_0$. Now we prove the converse, i.e, the result of orthogonal projection method onto $\mathcal{K}$ starting with $x_0$ minimizes A-norm of error over $x_0+\mathcal{K}$. We know $V^T(b-A\tilde{x}) = 0$, i.e, $(x_*-\tilde{x},v)_A=0\;\forall\;v\in\mathcal{K}$. 
\begin{align*}
\implies \|x_*-x\|_A &= \|x_*-\tilde{x}+\tilde{x}-x\|_A,\qquad(\tilde{x},x\in x_0+\mathcal{K}) \\
&= \|x_*-\tilde{x}\|_A+\|\tilde{x}-x\|_A,\,\text{(since }x_*-\tilde{x}\text{ is A-orthogonal to }\mathcal{K}) \\
\implies \|x_*-\tilde{x}\|_A &\le \|x_*-x\|_A,\;\forall\;x\in x_0+\mathcal{K}.
\end{align*}
Therefore $\tilde{x}$ minimizes the A-norm of the error.
\end{proof}

\begin{corollary}
Let A be an arbitrary square matrix and assume that $\mathcal{L}=A\mathcal{K}.$ Then a vector $\tilde{x}$ is the result of an (oblique) projection method onto $\mathcal{K}$ orthogonally to $\mathcal{L}$ with the starting vector $x_0$ iff it minimizes the 2-norm of the residual vector $b-Ax$ over $x\in x_0+\mathcal{K}$, i.e, iff
$$\tilde{x}=\underset{x\in x_0+\mathcal{K}}{\arg\min}\|b-Ax\|_2$$
\end{corollary}

\begin{prop}
Let $\tilde{x}$ be the approximate solution obtained from a projection process onto $\mathcal{K}$ orthogonally to $\mathcal{L}=A\mathcal{K}$, and let $\tilde{r} = b-A\tilde{x}$. Then,
$$\tilde{r}=(I-P)r_0,$$
where P denotes the orthogonal projector onto $\mathcal{K}$.
\end{prop}

\begin{proof}
Let $r_0=b-Ax_0$, then
\begin{align*}
\tilde{r} &= b-A\tilde{x}\\
&= b-A(x_0+\delta),\qquad(\delta\in\mathcal{K})\\
&= r_0-A\delta.
\end{align*}
By orthogonality condition we have $\tilde{r}\perp A\mathcal{K}$, i.e, $A\delta$ is the projection of $r_0$ onto $A\mathcal{K}$. Therefore, if P is the orthogonal projector onto $A\mathcal{K}$, then
$$Pr_0=A\delta \implies \tilde{r}=(I-P)r_0$$
It follows from the above that $\Vert \tilde{r}\Vert_2 \le \Vert r_0\Vert_2.$ Therefore, this class of methods can be termed as \textbf{Residual Projection Methods}.
\end{proof}

\begin{prop}
Let $\tilde{x}$ be the approximate solution obtained from an orthogonal projection process onto $\mathcal{K}$, and let $\tilde{d} = x_*-\tilde{x}$. Then,
$$\tilde{d}=(I-P_A)d_0,$$
where $P_A$ denotes the projector onto $\mathcal{K}$, which is orthogonal with respect to A-inner product.
\end{prop}

\begin{proof}
Let $d_0=x_*-x_0$ be the initial error, and let $\tilde{d}=x_*-\tilde{x}$, where $\tilde{x}=x_0+\delta$ is the approximate solution resulting from the projection step. We know that residual of the approximate solution is orthogonal to $\mathcal{K}$, i.e, $\tilde{r}=A\tilde{d}=A(d_0-\delta)$, $\tilde{r}\perp \mathcal{K}$.
\begin{align*}
&\implies (A(d_0-\delta),w)=0\;\forall\;w\in\mathcal{K} \\
&\implies (d_0-\delta,w)_A=0\;\forall\;w\in\mathcal{K}
\end{align*}
Therefore, if $P_A$ is the projector onto $A\mathcal{K}$, which is orthogonal with respect to A-inner product, then $\delta$ is the A-orthogonal projection of $d_0$, i.e,
$$P_Ad_0=\delta \implies \tilde{d}=(I-P_A)d_0.$$
It follows from the above that $\Vert \tilde{d}\Vert_A \le \Vert d_0\Vert_A.$ Therefore, this class of methods can be termed as \textbf{Error Projection Methods}.
\end{proof}


Define $\mathcal{P}_{\mathcal{K}}$ to be the orthogonal projector onto $\mathcal{K}$ and let $\mathcal{Q}^\mathcal{L}_\mathcal{K}$ be the (oblique) projector onto $\mathcal{K}$ and orthogonally to $\mathcal{L}$. Then 
$$\mathcal{P}_{\mathcal{K}}x\in\mathcal{K}\;and\;x-\mathcal{P}_{\mathcal{K}}x\perp\mathcal{K},$$
$$\mathcal{Q}^\mathcal{L}_\mathcal{K}x\in\mathcal{K}\;and\;x-\mathcal{Q}^\mathcal{L}_\mathcal{K}x\perp\mathcal{L}$$

\begin{theorem}
Assume that $\mathcal{K}$ is invariant under A and the initial residue, i.e, $r_0=b-Ax_0$ belongs to $\mathcal{K}.$ Then the approximate solution obtained from any (oblique or orthogonal) projectioon method onto $\mathcal{K}$ is exact.
\end{theorem}

\begin{proof}
An approximate solution $\tilde{x}$ is defined by 
\begin{align*}
&\mathcal{Q}^\mathcal{L}_\mathcal{K}(b-A\tilde{x})=0,\;where\;\tilde{x}=x_0+\delta,\;\delta\in\mathcal{K}. \\
&\implies \mathcal{Q}^\mathcal{L}_\mathcal{K}(b-Ax_0-A\delta)=0 \\
&\implies \mathcal{Q}^\mathcal{L}_\mathcal{K}r_0=\mathcal{Q}^\mathcal{L}_\mathcal{K}A\delta \\
&\text{But }\mathcal{K}\text{ is invariant under A, then }A\delta\in\mathcal{K}. \\
&\implies r_0=A\delta,\;(\text{since }r_0\in\mathcal{K}\text{ and }\mathcal{Q}^\mathcal{L}_\mathcal{K}A\delta=A\delta) \\
&\implies A\tilde{x}=b
\end{align*}
\end{proof}

\begin{theorem}[\textbf{General Error Bound}]
Let $\gamma=\|\mathcal{Q}^\mathcal{L}_\mathcal{K}A(I-\mathcal{P}_\mathcal{K})\|_2$ and assume that b is a member of $\mathcal{K}$ and $x_0=0$. Then the exact solution $x_*$ of the problem is such that
$$\|b-\mathcal{Q}^\mathcal{L}_\mathcal{K}A\mathcal{P}_\mathcal{K}x_*\|_2\le\gamma\|(I-\mathcal{P}_\mathcal{K})x_*\|_2.$$
\end{theorem}

\begin{proof}
Since $b\in\mathcal{K},$
\begin{align*}
b-\mathcal{Q}^\mathcal{L}_\mathcal{K}A\mathcal{P}_\mathcal{K}x_*&=\mathcal{Q}^\mathcal{L}_\mathcal{K}b-\mathcal{Q}^\mathcal{L}_\mathcal{K}A\mathcal{P}_\mathcal{K}x_* \\
&=\mathcal{Q}^\mathcal{L}_\mathcal{K}(b-A\mathcal{P}_\mathcal{K}x_*) \\
&=\mathcal{Q}^\mathcal{L}_\mathcal{K}A(I-\mathcal{P}_\mathcal{K})x_* \\
&=\mathcal{Q}^\mathcal{L}_\mathcal{K}A(I-\mathcal{P}_\mathcal{K})(I-\mathcal{P}_\mathcal{K})x_* \\
\implies \|b-\mathcal{Q}^\mathcal{L}_\mathcal{K}A\mathcal{P}_\mathcal{K}x_*\|_2 &= \|\mathcal{Q}^\mathcal{L}_\mathcal{K}A(I-\mathcal{P}_\mathcal{K})(I-\mathcal{P}_\mathcal{K})x_*\|_2 \\
&\le \|\mathcal{Q}^\mathcal{L}_\mathcal{K}A(I-\mathcal{P}_\mathcal{K})\|_2\|(I-\mathcal{P}_\mathcal{K})x_*\|_2 \\
\implies \|b-\mathcal{Q}^\mathcal{L}_\mathcal{K}A\mathcal{P}_\mathcal{K}x_*\|_2 &\le \gamma\|(I-\mathcal{P}_\mathcal{K})x_*\|_2 \\
\end{align*}
\end{proof}

\section{One-Dimensional Projection Methods}

One-dimensional projection processes are defined when $\mathcal{K}=span\{v\}\;and\\\;\mathcal{L}=span\{w\}.$ In this case, the new approximation  takes the form $x\leftarrow x+\alpha v$, where the orthogonality condition $r-A\delta\perp w$ yields,
$$\alpha = \frac{(r,w)}{(Av,w)},\;where\;r=b-Ax_0.$$

\subsection{Steepest Descent}

The steepest descent algorithm is defined when A is SPD and $v=w=r.$ 

\begin{lemma}[\textbf{Kantorovich inequality}]
Let B be any real SPD matrix and $\lambda_1,\;\lambda_n$ its largest and smallest eigenvalues. Then,
$$\frac{(Bx,x)(B^{-1}x,x)}{(x,x)}\le\frac{(\lambda_1+\lambda_n)^2}{4\lambda_1\lambda_n},\;\forall\;x\neq 0$$
\end{lemma}

\begin{proof}
It is equvivalent to prove the statement for any unit vector x. Since B is SPD, it can be diagonalized by similarity transformation with an orthogonal matrix Q, $B=Q^TDQ.$
$$(Bx,x)(B^{-1}x,x)=(Q^TDQx,x)(Q^TD^{-1}Qx,x)=(DQx,Qx)(D^{-1}Qx,Qx).$$
Define $y=Qx=(y_1,y_2,\cdots,y_n)^T,\;and\;\beta_i={y_i}^2$. Then,
$$\lambda\equiv (Dy,y) = \sum^n_{i=1}\beta_i\lambda_i,\;\sum^n_{i=1}\beta_i=1$$
$$\psi(y)=(D^{-1}y,y)=\sum^n_{i=1}\beta_i\frac{1}{\lambda_i}.$$
Note that $\lambda$ is a convex combinations of eigenvalues of B. Then,
$$(Bx,x)(B^{-1}x,x)=\lambda\psi(y).$$
Noting that $f(\lambda)=1/{\lambda}$ is a convex function for $x\in\mathbb{R}_{++}$, $\psi(y)$ containis all the convex combinations of $1/\lambda_i$s which is bounded above by line passing through $(\lambda_1,1/{\lambda_1})\;and\;(\lambda_n,1/{\lambda_n})$, i.e,
$$\psi(y)\le \frac{1}{\lambda_1}+\frac{1}{\lambda_n}-\frac{\lambda}{\lambda_1\lambda_n}.$$
$$\implies (Bx,x)(B^{-1}x,x)=\lambda\psi(y)\le \lambda\Big(\frac{1}{\lambda_1}+\frac{1}{\lambda_n}-\frac{\lambda}{\lambda_1\lambda_n}\Big).$$
The right-hand side is maximum when $\lambda=\dfrac{\lambda_1+\lambda_n}{2}$  yielding,
$$(Bx,x)(B^{-1}x,x)\le\frac{(\lambda_1+\lambda_n)^2}{4\lambda_1\lambda_n}$$
\end{proof}

\begin{algorithm}
\caption{Steepest Descent Algorithm}
\begin{algorithmic}[1]
\State Compute $r=b-Ax$ and $p=Ar$
\Repeat
	\State $\alpha\gets (r,r)/(p,r)$
	\State $x\gets x+\alpha r$
	\State $r\gets r-\alpha p$
	\State Compute $p=Ar$
\Until{Convergence}
\end{algorithmic}
\end{algorithm}

\begin{theorem}
Let A be a SPD. Then, A-norms of the error vectors $d_k=x_*-x_k$ generated by the above algorithm satsify the following relation:
$$\|d_{k+1}\|_A\le\Big(\frac{\lambda_1-\lambda_n}{\lambda_1+\lambda_n}\Big)\|d_k\|_A,$$
and the algorithm converges for any initial guess $x_0.$
\end{theorem}

\begin{proof}
We know that $d_{k+1}=x_*-x_{k+1}$, but $x_{k+1}=x_k+\alpha_k r_k.$
$$\implies d_{k+1}=x_*-(x_k+\alpha_k r_k)=d_k-\alpha_k r_k.$$
Now consider,
\begin{align*}
\|d_{k+1}\|^2_A &= (d_{k+1},d_k-\alpha_k r_k)_A \\
&= (d_{k+1},d_k)_A-(d_{k+1},\alpha_k r_k)_A \\
(d_{k+1},\alpha_k r_k)_A &= (Ad_{k+1},\alpha_k r_k) = (r_{k+1},\alpha_k r_k), \\
&= (r_k-\alpha_kAr_k,r_k),\text{ where }\alpha_k=\frac{(r_k,r_k)}{(Ar_k,r_k)}, \\
&= (r_k,r_k)-\frac{(r_k,r_k)}{(Ar_k,r_k)}(Ar_k,r_k) = 0 = (r_{k+1},r_k). \\
\implies (d_{k+1},\alpha_k r_k)_A &= 0.
\end{align*}
\begin{align*}
\implies \|d_{k+1}\|^2_A &= (d_{k+1},d_k)_A \\
&= (d_{k+1},Ad_k) \qquad(\text{since A is SPD}),\\
&= (d_k-\alpha_k r_k,r_k) \\
&= (A^{-1}r_k,r_k)-\alpha_k(r_k,r_k) \\
\text{But, }\|d_k\|^2_A = (Ad_k,d_k) &= (r_k,d_k) = (A^{-1}r_k,r_k), \\
\implies \|d_{k+1}\|^2_A &= (A^{-1}r_k,r_k)\Big(1-\frac{(r_k,r_k)^2}{(Ar_k,r_k)(A^{-1}r_k,r_k)}\Big), \\
\text{From Kantorovich inequality,} \\
&\le \|d_{k}\|^2_A\Big(1-\frac{4\lambda_1\lambda_n}{(\lambda_1+\lambda_n)^2}\Big), \\
\implies \|d_{k+1}\|_A&\le\Big(\frac{\lambda_1-\lambda_n}{\lambda_1+\lambda_n}\Big)\|d_k\|_A.
\end{align*}
\end{proof}

\section{Krylov Subspace Methods}

We define Krylov Subspace to be 
$$\mathcal{K}_m(A,v)=span\{v,Av,A^2v,\cdots,A^{m-1}v\}.$$
Then, $x=p(A)v,\;\forall\;x\in\mathcal{K}_m\text{, where }deg(p)<m.$

\begin{mydef}[Minimal Polynomial of a vector]
Monic polynomial of least degree such that $p(A)v=0$ is called minimal polynomial of v and degree of such polynomial is called grade$(\mu)$.
\end{mydef}

\begin{theorem}
Let $\mu$ be the grade of v. Then $\mathcal{K}_\mu$ is invariant under A and $\mathcal{K}_\mu = \mathcal{K}_m\;\forall\;m\ge\mu.$
\end{theorem}

\begin{proof}
Since, grade of v is $\mu$ there exists a polynomial p of degree $\mu$, such that $p(A)v=0,\;where\;p(A) = p_0I+p_1A+\cdots+p_{\mu-1}A^{\mu-1}+A^\mu.$
$$\implies A^\mu v = -(p_0I+p_1A+\cdots+p_{\mu-1}A^{\mu-1})v\qquad(1)$$
But, $\forall\;x\in\mathcal{K}_\mu,\;x=q(A)v,\;deg(q)<\mu,$ i.e,
\begin{align*}
x &= q_0v+q_1Av+\cdots+q_{\mu-1}A^{\mu-1}v,\;\forall\;x\in\mathcal{K}_\mu, \\
\implies Ax &= q_0Av+q_1A^2v+\cdots+q_{\mu-1}A^\mu v, \\
\text{Case 1: }& q_{\mu-1} = 0,\;then\;Ax\in\mathcal{K}_\mu. \\
\text{Case 2: }& q_{\mu-1} \neq 0,\text{ then replace }A^\mu v\;by\;(1),\;Ax\in\mathcal{K}_\mu.
\end{align*}
Therefore, $\mathcal{K}_\mu$ is invariant under A. Similarily it can be seen that $\mathcal{K}_\mu = \mathcal{K}_m \\
\;\forall\;m\ge\mu.$
\end{proof}

\begin{corollary}
$dim(\mathcal{K}_m) = min\{m,grade(v)\}.$
\end{corollary}

\newpage

\section{Arnoldi's Method for Linear Systems (FOM)}

Arnoldi's procedure is an algorithm for building an orthogonal basis of the Krylov subspace $\mathcal{K}_m$.

\begin{algorithm}
\caption{Arnoldi-Modified Gram-Schmidt}
\begin{algorithmic}[1]
\State Choose a vector $v_1$ of norm 1
\For{$j= 1,2,\cdots,m$}
	\State Compute $w_j = Av_j$
	\For{$i= 1,2,\cdots,j$}
		\State $h_{ij} = (w_j,v_i)$
		\State $w_j = w_j - h_{ij}v_i$
	\EndFor
	\State EndDo
	\State $h_{j+1,j} = \|w_j\|_2$. 
	\State If $h_{j+1,j}=0$ Stop; found an invariant subspace $[v_1,\cdots,v_j]$
	\State $v_{j+1}=w_j/h_{j+1,j}$
\EndFor
\State EndDo
\end{algorithmic}
\end{algorithm}

\begin{prop}
Denote by $V_m=[v_1,v_2,\cdots,v_m]_{n\times m}\;and\;\bar{H}_m,\;the\;(m+1)\times m$ Hessenberg matrix whose non-zero entries $h_{ij}$ are defined by the above algorithm and by $H_m$ the matrix obtained from $\bar{H}_m$ by removing the last row. Then,
$$AV_m=V_mH_m+w_me^T_m=V_{m+1}\bar{H}_m,$$
$$V^T_mAV_m=H_m.$$
\end{prop}

\begin{proof}
From lines 6,8 we have, $w_j = Av_j - h_{ij}v_i$ and $w_j=v_{j+1}h_{j+1,j}.$
$$\implies Av_j = \sum^{j+1}_{i=1}h_{ij}v_i \implies AV_m = V_mH_m+w_me^T_m=V_{m+1}\bar{H}_m.$$
$$\text{Since }V^T_m\text{ is orthogonal, we get }V^T_mAV_m=H_m.$$
\end{proof}

Given an initial guess $x_0$ to the original linear system $Ax=b$, we now consider an orthogonal projection method which takes $\mathcal{L}=\mathcal{K}=\mathcal{K}_m(A,r_0),$ with
$$\mathcal{K}_m(A,r_0)=span\{r_0,Ar_0,A^2r_0,\cdots,A^{m-1}r_0\},$$
in which $r_0=b-Ax_0.$ This method seeks an approximate solution $x_m$ from the affine subspace $x_0+\mathcal{K}_m$ of dimension m by imposing the following orthogonality constraint:
$$b-Ax_m\perp \mathcal{K}_m.$$
If $v_1=r_0/\|r_0\|_2$ in Arnoldi's method, and we set $\beta=\|r_0\|_2,$ then
$$V^T_mAV_m=H_m,\;V^T_mr_0=V^T_m(\beta v_1)=\beta e_1.$$
As a result, the approximate solution using the above m-dimensional subspaces is given by:
$$x_m=x_0+V_my_m,$$
where $y_m$ can be found by imposing orthogonality constraint that $$V^T_m(b-Ax_m)=0\implies y_m=H^{-1}_m(\beta e_1).$$

\newpage

\begin{algorithm}
\caption{Full Orthogonalization Method (FOM)}
\begin{algorithmic}[1]
\State Compute $r_0=b-Ax_0,\;\beta=\|r_0\|_2,\;and\;v_1=r_0/\beta$
\State Define the $m\times m$ matrix $H_m = \{h_{ij}\}_{i,j=1,2,\cdots,m};Set\;H_m=0$
\For{$j= 1,2,\cdots,m$}
	\State Compute $w_j = Av_j$
	\For{$i= 1,2,\cdots,j$}
		\State $h_{ij} = (w_j,v_i)$
		\State $w_j = w_j - h_{ij}v_i$
	\EndFor
	\State EndDo
	\State $h_{j+1,j} = \|w_j\|_2$. If $h_{j+1,j}=0$ Stop
	\State $v_{j+1}=w_j/h_{j+1,j}$
\EndFor
\State EndDo
\State Compute $y_m=H^{-1}_m\beta e_1$ and $x_m=x_0+V_my_m$
\end{algorithmic}
\end{algorithm}

\begin{prop}
The residual vector of the approximate solution $x_m$ computed by the FOM Algorithm is such that 
$$b-Ax_m=-h_{m+1,m}e^T_my_mv_{m+1}$$
and, therefore,
$$\|b-Ax_m\|_2=h_{m+1,m}|e^T_my_m|.$$

\begin{proof}
\begin{align*}
b-Ax_m &= b-Ax_0-AV_my_m \\
&= r_0 - (V_mH_m+w_me^T_m)y_m \\
&= r_0 - V_mH_m(H^{-1}_m\beta e_1) - w_me^T_my_m \\
&= r_0 - V_mV^T_mr_0 - h_{m+1,m}e^T_my_mv_{m+1}=-h_{m+1,m}e^T_my_mv_{m+1}.
\end{align*}
\end{proof}
\end{prop}

\subsection{Variation 1: Restarted FOM}

\begin{algorithm}
\caption{Restarted FOM (FOM(m))}
\begin{algorithmic}[1]
\State Compute $r_0=b-Ax_0,\;\beta=\|r_0\|_2,\;and\;v_1=r_0/\beta$
\State Generate $V_m\;and\;H_m$ using Arnoldi algorithm starting with $v_1$.
\State Compute $y_m=H^{-1}_m\beta e_1$ and $x_m=x_0+V_my_m$. If satisfied then Stop.
\State Set $x_0=x_m$ and go to 1.
\end{algorithmic}
\end{algorithm}

\subsection{Variation 1: IOM and DIOM}

A formula can be developed whereby the current approximate solution $x_m$ can be computed from the previous approximation $x_{m-1}$ and a small number vectors are updated at each step. This progressive formulation of the solution leads to an algorithm termed as Direct IOM (DIOM).\\
The Hessenberg matrix obtained from IOM has a band structure with bandwidth $k+1$, i.e,

\begin{algorithm}
\caption{Incomplete Orthogonalization Method (IOM)}
\begin{algorithmic}[1]
\State Compute $r_0=b-Ax_0,\;\beta=\|r_0\|_2,\;and\;v_1=r_0/\beta$
\State Define the $m\times m$ matrix $H_m = \{h_{ij}\}_{i,j=1,2,\cdots,m};Set\;H_m=0$
\For{$j= 1,2,\cdots,m$}
	\State Compute $w_j = Av_j$
	\For{$i= max\{1,j-(k-1)\},2,\cdots,j$}
		\State $h_{ij} = (w_j,v_i)$
		\State $w_j = w_j - h_{ij}v_i$
	\EndFor
	\State EndDo
	\State $h_{j+1,j} = \|w_j\|_2$. If $h_{j+1,j}=0$ Stop
	\State $v_{j+1}=w_j/h_{j+1,j}$
\EndFor
\State EndDo
\State Compute $y_m=H^{-1}_m\beta e_1$ and $x_m=x_0+V_my_m$
\end{algorithmic}
\end{algorithm}


\begin{align*}
H_m &= \left( \begin{array}{ccccc}
h_{11} & h_{12} & h_{13} &  &  \\
h_{21} & h_{22} & h_{23} & h_{24} &  \\
 & h_{32} & h_{33} & h_{34} & h_{35} \\
 &  & h_{43} & h_{44} & h_{45} \\
 &  &  & h_{54} & h_{55} \\
\end{array} \right) = L_mU_m \\
&= \left( \begin{array}{ccccc}
1 &  &  &  &  \\
l_{21} & 1 &  &  &  \\
 & l_{32} & 1 &  &  \\
 &  & l_{43} & 1 &  \\
 &  &  & l_{54} & 1 \\
\end{array} \right)\times 
\left( \begin{array}{ccccc}
u_{11} & u_{12} & u_{13} &  &  \\
 & u_{22} & u_{23} & u_{24} &  \\
 &  & u_{33} & u_{34} & u_{35} \\
 &  &  & u_{44} & u_{45} \\
 &  &  &  & u_{55} \\
\end{array} \right) 
\end{align*}
The approximate solution then is given by 
$$x_m=x_0+V_mU^{-1}_mL^{-1}_m(\beta e_1).$$
Define $P_m\equiv V_mU^{-1}_m\;and\;z_m=L^{-1}_m(\beta e_1),$ we have $x_m=x_0+P_mz_m.$ Because of the structure of $U_m,\;P_m$ can be updated easily. Indeed, equating the last columns of the matrix relation $P_mU_m=V_m$ yields,
$$\sum^m_{i=m-k+1} u_{im}p_i=v_m\implies p_m = \frac{1}{u_{mm}}\Bigg( v_m-\sum^{m-1}_{i=m-k+1}u_{im}p_i \Bigg).$$
Therefore, $p_m$ can be computed using previous $p_i's$ and $v_m$. In addition, due to the structure of $L_m$, we have compute $z_m$ by,
$$z_m=\left[ \begin{array}{c} z_{m-1} \\ \zeta_m \end{array} \right]\text{, where }\zeta_m=-l_{m,m-1}\zeta_{m-1}.$$
Now, the approximate solution is,
$$x_m=x_0+\left[ \begin{array}{cc} P_{m-1} & p_m \end{array} \right]\left[ \begin{array}{c} z_{m-1} \\ \zeta_m \end{array} \right]=x_0+P_{m-1}z_{m-1}+p_m\zeta_m.$$
Noting that $x_{m-1}=P_{m-1}z_{m-1}$, $x_m$ can be updated as follows:
$$x_m=x_{m-1}+\zeta_mp_m.$$
This gives the following algorithm, called \textbf{Incomplete Orthogonalization Method}(DIOM).

\begin{algorithm}
\caption{Direct Incomplete Orthogonalization Method (DIOM)}
\begin{algorithmic}[1]
\State Choose $x_0$ and compute $r_0=b-Ax_0,\;\beta=\|r_0\|_2,\;and\;v_1=r_0/\beta$
\For{$m= 1,2,\cdots$, until convergence}
	\State Compute $w_m = Av_m$
	\For{$i= max\{1,m-k+1\},2,\cdots,m$}
		\State $h_{im} = (w_m,v_i)$
		\State $w_m = w_m - h_{im}v_i$
	\EndFor
	\State $h_{m+1,m} = \|w_m\|_2$. If $h_{m+1,m}=0$ Stop
	\State $v_{m+1}=w_m/h_{m+1,m}$
	\State Update the LU factorization of $H_m$, i.e, obtain the last column 	
	\State $\qquad U_m$ using the previous k pivots. If $u_{mm}=0$ Stop.
	\State $\zeta_m = \beta$ if $m=1$ else $-l_{m,m-1}\zeta_{m-1}$
	\State $p_m = u^{-1}_{mm}\Big( v_m-\sum^{m-1}_{i=m-k+1}u_{im}p_i \Big)(\text{for }i\le0\text{ set }u_{im}p_i\equiv0)$
	\State $x_m=x_{m-1}+\zeta_mp_m$
\EndFor
\State EndDo
\end{algorithmic}
\end{algorithm}

\begin{remark}
Observe that $V^T_mAV_m=H_m$ is still valid because the orthogonality properties were not used to derive this relation. As a consequence the following result  is also valid,
\begin{align*}
b-Ax_m &= -h_{m+1,m}e^T_my_mv_{m+1} \\
\implies \|b-Ax_m\|_2 &= h_{m+1,m}|e^T_my_m| \\
\text{But, }y_m = H^{-1}_m(\beta	e_1) &= U^{-1}_mz_m \implies e^T_my_m = \zeta_m/u_{mm} \\
\implies \|b-Ax_m\|_2 &= h_{m+1,m}\Bigl|\frac{\zeta_m}{u_{mm}}\Bigr|
\end{align*}
\end{remark}

Since the residual vectors is a scalar multiple of $v_{m+1}$ and since the $v_i$'s are no longer orthogonal, IOM and DIOM are not orthogonal projection techniques. They can however be viewed as oblique projection techniques onto $\mathcal{K}_m$ orthogonally to an artificially constructed subspace.

\begin{prop}
IOM and DIOM are mathematically equivalent to projection process onto $\mathcal{K}_m$ and orthogonally to
$$\mathcal{L}_m=span\{z_1,z_2,\cdots,z_m\},$$
$$where\;z_i=v_i-(v_i,v_{m+1})v_{m+1},\;i=1,2,\cdots,m.$$
\end{prop}

\begin{proof}
From the construction of $\mathcal{L}_m,\;v_{m+1}$ is orthogonal to $\mathcal{L}_m$ and we know the final residue $r_m$ is a scalar multiple of $v_{m+1}$, hence the approximate solution $x_m\in\mathcal{K}_m$ and residue vector $r_m\perp\mathcal{L}_m$.  
\end{proof}

\section{Symmetric Lanczos Algorithm}

The symmetric lanczos algorithm can be viewed as a simplification of Arnoldi's method for the particular case of symmetric matrix. When A is symmetric, then the Hessenberg matrix $H_m$ will become symmetric tridiagonal. The standard notation used to descsribe the Lanczos algorithm is obtained by setting 
$$\alpha_j=h_{jj},\;\beta_j=h_{j-1,j},$$
and if $T_m$ denotes the resulting $H_m$ matrix, it is of the form,
$$
T_m = \left( \begin{array}{ccccc}
\alpha_1 & \beta_2 &  &  &  \\
\beta_2 & \alpha_2 & \beta_3 &  &  \\
 & . & . & . &  \\
 &  & \beta_{m-1} & \alpha_{m-1} & \beta_{m} \\
 &  &  & \beta_m & \alpha_m \\
\end{array} \right). 
$$

This leads to the following form of Modified Gram-Schmidt variant of \\ Arnoldi's method:

\begin{algorithm}
\caption{Lanczos Method for Linear Systems}
\begin{algorithmic}[1]
\State Compute $r_0=b-Ax_0,\;\beta=\|r_0\|_2,\;and\;v_1=r_0/\beta$
\State Set $\beta_1=0\;and\;v_0=0$
\For{$j= 1,2,\cdots,m$} \Comment Orthogonalization Procedure
	\State $w_j = Av_j-\beta_jv_{j-1}$
	\State $\alpha_j=(w_j,v_j)$
	\State $w_j = w_j-\alpha_jv_j$
	\State $\beta_{j+1} = \|w_j\|_2.$ If $\beta_{j+1}=0$ then Stop
	\State $v_{j+1} = w_j/\beta_{j+1}$
\EndFor
\State EndDo
\State Set $T_m=tridiag(\beta_i,\alpha_i,\beta_{i+1})$, and $V_m=[v_1,\cdots,v_m].$
\State Compute $y_m=H^{-1}_m\beta e_1$ and $x_m=x_0+V_my_m$
\end{algorithmic}
\end{algorithm}

\section{Conjugate Gradient}

The conjugate gradient algorithm can be derived from the Lanczos algorithm in the same way DIOM was derived from IOM. Infact, the conjugate gradient algorithmm can be viewed as a variation of DIOM for the case when A is symmetric.

First write the LU factorization of $T_m$ as $T_m=L_mU_m.$ The matrix $L_m$ is unit lower bidiagonal and $U_m$ is unit upper bidiagonal matrix. Thus the factorization of $T_m$ is of the form
$$
T_m = 
\left( \begin{array}{ccccc}
1 &  &  &  &  \\
\lambda_2 & 1 &  &  &  \\
 & . & . &  &  \\
 &  & \lambda_{m-1} & 1 &  \\
 &  &  & \lambda_m & 1 \\
\end{array} \right)\times 
\left( \begin{array}{ccccc}
\eta_1 & \beta_2 &  &  &  \\
 & \eta_2 & \beta_3 &  &  \\
 &  & . & . &  \\
 &  &  & \eta_{m-1} & \beta_m \\
 &  &  &  & \eta_m \\
\end{array} \right).
$$

The approximate solution is then given by,
$$x_m=x_0+V_mU^{-1}_mL^{-1}_m(\beta e_1) = x_0+P_mz_m.$$
As for DIOM, $p_m$, the last column of $P_m$, can be computed from the previous $p_i's\;and\;v_m$ by the simple update
$$p_m=\eta^{-1}_m[v_m-\beta_mp_{m-1}].$$
Note that $\beta_m$ is a scalar computed from the Lanczos algorithm, while $\eta_m$ results from the m-th Gaussian elimination step on the tridiagonal matrix, i.e, $\lambda_m=\frac{\beta_m}{\eta_{m-1}},\;\eta_m=\alpha_m-\lambda_m\beta_m.$ In addition, following again what has been shown for DIOM, $z_m=\left[ \begin{array}{c} z_{m-1} \\ \zeta_m \end{array} \right],\;where\;\zeta_m=-\lambda_m\zeta_{m-1}.$
As a result, $x_m$ can be updated at each step as follows:
$$x_m=x_{m-1}+\zeta_mp_m.$$
This gives the following algorithm, which we call as direct version of Lanczos algorithm for linear systems.

\begin{algorithm}
\caption{D-Lanczos}
\begin{algorithmic}[1]
\State Choose $x_0$ and compute $r_0=b-Ax_0,\;\zeta_1=\beta=\|r_0\|_2,\;and\;v_1=r_0/\beta$
\State $\lambda_1=\beta_1=0,\;p_0=0$
\For{$m= 1,2,\cdots$, until convergence}
	\State Compute $w_m = Av_m-\beta_mv_{m-1}\;and\;\alpha_m=(w,v_m)$
	\State If $m > 1$ then compute $\lambda_m=\frac{\beta_m}{\eta_{m-1}}$ and $\zeta_m=-\lambda_m\zeta_{m-1}$
	\State $\eta_m=\alpha_m-\lambda_m\beta_m$
	\State $p_m = \eta^{-1}_m[v_m-\beta_mp_{m-1}]$
	\State $x_m=x_{m-1}+\zeta_mp_m$
	\State If $x_m$ has converged then Stop
	\State $w=w-\alpha_mv_m$
	\State $\beta_{m+1}=\|w\|_2,\;v_{m+1}=w/\beta_{m+1}$
\EndFor
\State EndDo
\end{algorithmic}
\end{algorithm}

Observe that the residual vector for this algorithm is in the direction of $v_{m+1}.$ Therefore, the residual vectors are orthogonal to each other as in FOM. Likewise, the vectors $p_i$ are A-orhogonal or conjugate to each other.

\begin{prop}
Let $r_m=b-Ax_m,\;and\;p_m,m=0,1,\cdots,$ be the residual vectors and auxiliary vectors produced by D-Lanczos algorithm. Then,
\begin{enumerate}[i)]
\item Each residual vector $r_m$ is such that $r_m=\sigma_{m}v_{m+1}$, where $\sigma_m$ is a certain scalar. As a result, the residual vectors are orthogonal to each other.
\item The auxiliary vectors $p_i$ form an A-conjuagte set, i.e,
$$(Ap_i,p_j)=0,\;for\;i\neq j.$$
\end{enumerate}
\end{prop}

\begin{proof}
The first part is immediate consequence of the following relation:
$$r_m=b-Ax_m=-h_{m+1,m}e^T_my_mv_{m+1}$$
Now we prove the second part. Consider the matrix $P^T_mAP_m$,
$$P^T_mAP_m = U^{-T}_mV^T_mAV_mU^{-1}_m = U^{-T}_mT_mU^{-1}_m = U^{-T}_mL_m$$
Now we observe $U^{-T}_mL_m$ is a lower triangular matrix which is also symmetric since it is equal to the symmetric matrix $P^T_mAP_m$. Therefore it must be diagonal.
\end{proof}

A consequence of the above proposition is that a version of the algorithm  can be derived by imposing the orthogonality and conjugacy conditions. This gives the Conguate Gradient algorithm which we now derive. 

\begin{remark}
Inorder to conform with the standard notation used in the literature to describe the algorithm, the indexing of the p vectors now begins at zero instead of one as was done so far.
\end{remark}

The vector $x_{j+1},r_{j+1},p_{j+1}$ can be expressed as follows:
$$x_{j+1}=x_j+\alpha_jp_j,\quad r_{j+1}=r_j-\alpha_jAp_j,\quad p_{j+1}=r_{j+1}+\beta_jp_j.$$
It can be seen that $(Ap_j,r_j)=(Ap_j,p_j+\beta_jp_{j-1})=(Ap_j,p_j).$ Now imposing constraints,
\begin{enumerate}
\item Orthogonality Conditions on $r_i$'s gives $(r_j-\alpha_jAp_j,r_j)=0$, as a result,
$$\alpha_j = \frac{(r_j,r_j)}{(Ap_j,r_j)}=\frac{(r_j,r_j)}{(Ap_j,p_j)}$$
\item Conjugacy Conditions on $p_i$'s gives $(p_{j+1},Ap_j)=0$. But,
\begin{align*}
(p_{j+1},Ap_j) &= (r_{j+1}+\beta_jp_j,Ap_j) = -\frac{1}{\alpha_j}(r_{j+1}+\beta_jp_j,r_{j+1}-r_j)\\ 
(p_{j+1},&Ap_j) = 0 \implies \beta_j=\frac{(r_{j+1},r_{j+1})}{(r_j,r_j)}
\end{align*}
\end{enumerate}
It is important to note that the scalar $\alpha_j,\beta_j$ in this algorithm and D-Lanczos are different.

Putting these relation together gives the following algorithm.

\begin{algorithm}
\caption{Conjugate Gradient}
\begin{algorithmic}[1]
\State Choose $x_0$ and compute $r_0=b-Ax_0,\;p_0=r_0.$
\For{$j= 1,2,\cdots$, until convergence}
	\State $\alpha_j=(r_j,r_j)/(Ap_j,p_j)$
	\State $x_{j+1}=x_j+\alpha_jp_j$
	\State $r_{j+1}=r_j-\alpha_jAp_j$
	\State $\beta_j=(r_{j+1},r_{j+1})/(r_j,r_j)$
	\State $p_{j+1}=r_{j+1}+\beta_jp_j$
\EndFor
\State EndDo
\end{algorithmic}
\end{algorithm}

\newpage

\section{Convergence Analysis}

Here we show the convergence of CG Algorithm using Chebyshev polynomials.

\subsection{Real Chebyshev Polynomials}

\begin{mydef}
The Chebyshev polynomial of the first kind of degree k is defined by:
\[   
C_k(t) = 
     \begin{cases}	
      \, cos[k\,cos^{-1}(t)], &\quad\text{for }|t|\le 1\\
       cosh[k\,cosh^{-1}(t)], &\quad\text{otherwise.}\\
     \end{cases}
\]
\end{mydef}

It can be seen that $C_0(t)=1,\;C_1(t)=t,$ can be easily extended by the following trigonometric relation
$$cos[(k+1)\theta]+cos[(k-1)\theta]=2cos\theta\,cosk\theta.$$
This also shows the important three-term recurrence relation
$$C_{k+1}(t)=2tC_k(t)-C_{k-1}(t).$$

\begin{prop}
If $|t|>1$, then $C_k(t)=\frac{1}{2}\bigg[\Big(t+\sqrt{t^2-1}\Big)^k+\Big(t+\sqrt{t^2-1}\Big)^{-k}\bigg]$
\end{prop}

\begin{proof}
$\text{We know, }C_k(t) = cosh[k\,cosh^{-1}(t)],|t|>1,\;cosh\,\theta=\frac{1}{2}\big(e^\theta+e^{-\theta}\big).$ Then,
$C_k(t) = \frac{1}{2}\Big[(e^\theta)^k+(e^\theta)^{-k}\Big],$ where $\theta = cosh^{-1}(t),$ i.e, $t=cosh\,\theta$ 
\begin{align*}
&\implies t = \frac{1}{2}\big(e^\theta+e^{-\theta}\big) \implies e^{2\theta}-2te^\theta+1=0 \\
&\implies e^\theta = t \pm \sqrt{t^2-1}
\end{align*}
Notice that $t + \sqrt{t^2-1}=\frac{1}{t - \sqrt{t^2-1}}.$ Then, for $|t|>1,$
$$C_k(t)=\frac{1}{2}\bigg[\Big(t+\sqrt{t^2-1}\Big)^k+\Big(t+\sqrt{t^2-1}\Big)^{-k}\bigg]\gtrapprox\frac{1}{2}\Big(t+\sqrt{t^2-1}\Big)^k.$$
\end{proof}

In what follows  we denote by $\mathbb{P}_k$ as the set of all polynomials of degree k.

\begin{remark}
The Chebyshev polynomials are polynomials with the largest possible leading coefficient whose absolute value on the interval $[-1,1]$ is bounded by 1.
\end{remark}

\begin{theorem}
Let $[\alpha,\beta]$ be a non-empty interval in $\mathbb{R}$ and let $\gamma$ be any real scalar outside $[\alpha,\beta].$ Then,
$$\hat{C}_k(t)=\frac{C_k\Big(1+2\frac{t-\beta}{\beta-\alpha}\Big)}{C_k\Big(1+2\frac{\gamma-\beta}{\beta-\alpha}\Big)}=\underset{p\in\mathbb{P}_k,\,p(\gamma)=1}{\arg\min}\;\;\underset{t\in[\alpha,\beta]}{\max}|p(t)|.$$
\end{theorem}

\begin{remark}
If $\gamma\le\alpha$, the absolute values are needed in the denominators and exchanging the roles of $\alpha$ and $\beta$, i.e,
$$\hat{C}_k(t)=\frac{C_k\Big(1+2\frac{\alpha-t}{\beta-\alpha}\Big)}{C_k\Big(1+2\frac{\alpha-\gamma}{\beta-\alpha}\Big)}.$$
\end{remark}

We can further state following which is the result of the above remark and theorem.

\begin{corollary}
$\underset{p\in\mathbb{P}_k,\,p(\gamma)=1}{\arg\min}\;\;\underset{t\in[\alpha,\beta]}{\max}|p(t)| = \dfrac{1}{|C_k(1+2\frac{\gamma-\beta}{\beta-\alpha})|}$
\end{corollary}

\subsection{Convergence of CG Algorithm}

\begin{lemma}
Let $x_m$ be the approximate solution obtained from the m-th step of the CG algorithm, and let $d_m=x_*-x_m$ where $x_*$ is the exact solution. Then, $x_m=x_0+q_m(A)r_0$, where $q_m$ is a polynomial of degree $m-1$ such that
$$\|(I-Aq_m(A))d_0\|_A = \underset{q\in \mathbb{P}_{m-1}}{\min}\|(I-Aq(A))d_0\|_A.$$
\end{lemma}

\begin{proof}
Consider the following objective function in the polynomial q
$$\|(I-Aq(A))d_0\|_A = \|d_0-Aq(A)d_0\|_A = \|x_*-(x_0+Aq(A)r_0)\|_A = \|x_*-x\|_A.$$
And also observe that $\forall\,x\in x_0+\mathcal{K}_m(A,r_0),\,x=x_0+q(A)r_0,q\in\mathbb{P}_{m-1}$. Therefore minimizing x over $x_0+\mathcal{K}_m(A,r_0)$ and q over $\mathbb{P}_{m-1}$ are equivalent. We know that $x_m = \underset{x\in x_0+\mathcal{K}}{\arg\min}\|x_*-x\|_A$. $\text{Hence, }q_m=\underset{q\in \mathbb{P}_{m-1}}{\arg\min}\|(I-Aq(A))d_0\|_A.$
\end{proof}

\begin{theorem}
Let $\eta=\dfrac{\lambda_n}{\lambda_1-\lambda_n}$ and $\kappa = \dfrac{\lambda_1}{\lambda_n},$ then
$$\|x_*-x_m\|_A\le \dfrac{\|x_*-x_0\|_A}{C_m(1+2\eta)}\le 2\bigg[\dfrac{\sqrt{\kappa}-1}{\sqrt{\kappa}+1}\bigg]^m\|x_*-x_0\|_A.$$
\end{theorem}

\begin{proof}
From the previous lemma, it is known that $\|x_*-x_m\|_A$ is the minimum of the error over polynomials $r(t)=1-tq(t)$ which take the value one at 0, i.e,
$$\|x_*-x_m\|_A=\underset{r\in \mathbb{P}_m,\;r(0)=1}{\min}\|r(A)d_0\|_A.$$
Let $\lambda_i,i=1,2,\cdots,n$ are the eigenvalues of A, and $\xi_i,i=1,2,\cdots,n$ are the components of the initial error $d_0$ in the eigenbasis. Let eigen decomposition of $A=U\Lambda U^T,$ where $\Lambda=diag(\lambda_1,\lambda_2,\cdots,\lambda_n),$ then $d_0=U\xi,\;\xi=[\xi_1\;\xi_2\;\cdots\;\xi_n]^T$ and $r(A)=Ur(\Lambda)U^T$. Then,
\begin{align*}
\|r(A)d_0\|^2_A &= (Ar(A)d_0,r(A)d_0) = d^T_0r(A)A\,r(A)d_0 \\
&= \xi^TU^TUr(\Lambda)U^TU{\Lambda}U^TUr(\Lambda)U^T U\xi \\
&= \xi^T \Lambda r(\Lambda)^2\xi = \sum^n_{i=1}\lambda_ir(\lambda_i)^2\xi^2_i \le \underset{i}{\max}(r(\lambda_i)^2)\|d_0\|^2_A \\
\implies \|r(A)d_0\|^2_A &\le \underset{\lambda\in[\lambda_n,\lambda_1]}{\max}(r(\lambda_i)^2)\|d_0\|^2_A.
\end{align*}
Therefore,
\begin{align*}
\|x_*-x_m\|_A&\le\underset{r\in \mathbb{P}_m,\;r(0)=1}{\min}\;\;\underset{\lambda\in[\lambda_n,\lambda_1]}{\max}|r(\lambda_i)|\|d_0\|_A \\
&= \dfrac{\|x_*-x_0\|_A}{C_m(1+2\eta)}.
\end{align*}
We know, $C_m(t)\ge \frac{1}{2}\Big(t+\sqrt{t^2-1}\Big)^m$, then 
$$C_m(1+2\eta)\ge \frac{1}{2}\Big(1+2\eta+2\sqrt{\eta(\eta-1)}\Big)^m.$$
Now notice that
$$1+2\eta+2\sqrt{\eta(\eta-1)} = (\sqrt{\eta}+\sqrt{\eta+1})^2  = \dfrac{(\sqrt{\lambda_n}+\sqrt{\lambda_1})^2}{\lambda_1-\lambda_n} = \dfrac{\sqrt{\kappa}-1}{\sqrt{\kappa}+1}.$$
$$\text{Therefore, }\|x_*-x_m\|_A\le 2\bigg[\dfrac{\sqrt{\kappa}-1}{\sqrt{\kappa}+1}\bigg]^m\|x_*-x_0\|_A.$$
\end{proof}

\section{Deflated-CG Algorithm}

\subsection{Deflated Lanczos algorithm}

A $\in \mathbb{R}^{n \times n}$ is SPD and define $W=[w_1\;w_2\;\cdots\;w_k],\;w_i\in\mathbb{R}^n$ are linearly independent vectors. Let $v_1\in\mathbb{R}^n$ be a unit vector such that $W^Tv_1=0.$

\begin{prop}
Let W and A be defined as above, then $W^TAW$ is non-singular.
\end{prop}

\begin{proof}
FILL IN
\end{proof}

We want to generate a sequence $\{v_j\}_{j=1,2,\cdots}$ of vectors such that 
$$v_{j+1}\perp \text{span}\{W,v_1,v_2,\cdots,v_j\},\;\|v_{j+1}\|_2=1.$$

To obtain such a sequence, we apply Lanczos algorithm to the following auxiliary  matrix:
$$B=A-AW(W^TAW)^{-1}W^TA,$$
with the given initial vector $v_1$. Then we get a sequence $\{v_j\}_{j=1,2,\cdots}$ which satisfies
$$BV_j=V_jT_j+\sigma_{j+1}v_{j+1}e^T_j,\text{ where }
T_j = \left( \begin{array}{ccccc}
\rho_1 & \sigma_2 &  &  &  \\
\sigma_2 & \rho_2 & \sigma_3 &  &  \\
 & . & . & . &  \\
 &  & \sigma_{j-1} & \rho_{j-1} & \sigma_{j} \\
 &  &  & \sigma_j & \rho_j \\
\end{array} \right)$$

It is guaranteed by the Lanczos algorithm that the vectors $v_j$ are orthogonal to each other. By comparing the last column on the both sides of the above equation, we get
$$\sigma_{j+1}v_{j+1}=Bv_j-\sigma_jv_{j-1}-\rho_jv_j.$$

It can also be seen that $W^TB=0.$

\begin{prop}
Let W and $v_{j+1}$ be defined as above. Then $W^Tv_{j+1}=0\;\forall\;j=1,2,\cdots.$
\end{prop}

\begin{proof}
We prove by induction on j. It is clear from the definition that $W^Tv_1=0.$ Now assume that the propsition is true till $j-1$, i.e, $W^Tv_{k+1}=0\;\forall\;k=1,2,\cdots,j.$ Now we prove for $k=j+1.$ 

We know, $\sigma_{j+1}v_{j+1}=Bv_j-\sigma_jv_{j-1}-\rho_jv_j$, then
$$\implies \sigma_{j+1}W^Tv_{j+1} = W^TBv_j-\sigma_jW^Tv_{j-1}-\rho_jW^Tv_j=0.$$
\end{proof}

Therefore, we arrive at the following algorithm:

\begin{algorithm}
\caption{Deflated Lanczos Algorithm}
\begin{algorithmic}[1]
\State Choose k linearly independent vectors $w_1,w_2,\cdots,w_k$ and \\ \qquad define $W=[w_1\;w_2\;\cdots\;w_k]$
\State Choose an initial vector $v_1$ such that $W^Tv_1=0\text{ and }\|v_1\|_2=1.$
\For{$j= 1,2,\cdots,m$}
	\State Solve $W^TAW\hat{v_j}=W^TAv_j\text{ for }\hat{v_j}$
	\State $z_j=Av_j-AW\hat{v_j}$
	\State $\rho_j=v^T_jz_j$
	\State $\hat{v_{j+1}}=z_j-\sigma_jv_{j-1}-\rho_jv_j$
	\State $\sigma_{j+1}=\|\hat{v_{j+1}}\|_2$; If $\sigma_{j+1}=0$ Exit.
	\State $v_{j+1}=\hat{v_{j+1}}/\sigma_{j+1}$
\EndFor
\State EndDo
\end{algorithmic}
\end{algorithm}

\subsection{Deflated-CG algorithm}

We want to solve $Ax=b$, A is SPD. Based on deflated Lanczos algorithm we wish to derive a projection method similar to CG-algorithm. 

Assume $x_0$ is the initial guess and initial residue $r_0=b-Ax_0$ such that $r_0\perp W.$ Now set $v_1=r_0/\|r_0\|_2$ and define $\mathcal{K}_{k,j}(A,W,r_0)\equiv\text{span}\{W,V_j\}$. At j-th step of the projection, we seek for the approximate solution $x_j\in x_0+\mathcal{K}_{k,j}(A,W,r_0)$ and $r_j=b-Ax_j\perp \mathcal{K}_{k,j}(A,W,r_0).$

\begin{lemma}
If $x_j\;and\;r_j$ satisfy the above conditions, then $r_j=c_jv_{j+1},$ for some scalar $c_j$. Thus $\mathcal{K}_{k,j}(A,W,r_0)=\text{span}\{W,r_0,r_1,\cdots,r_{j-1}\}$ and the residuals $r_i$'s are orthogonal to each other.
\end{lemma}

\begin{proof}
Let $x_j=x_0+W\hat{\xi}_j+V_j\hat{\eta}_j$, for some $\hat{\xi}_j, \hat{\eta}_j.$ And we know $B=A-AW(W^TAW)^{-1}W^TA$ and $BV_j=V_jT_j+\sigma_{j+1}v_{j+1}e^T_j$, then we get $$AV_j=AW\Delta_j+V_jT_j+\sigma_{j+1}v_{j+1}e^T_j, \;\Delta_j=(W^TAW)^{-1}W^TA.$$
Hence, $r_j = r_0-AW\hat{\xi}_j-AV_j\hat{\eta}_j = r_0-AW\hat{\xi}_j-(AW\Delta_j+V_jT_j+\sigma_{j+1}v_{j+1}e^T_j)\hat{\eta}_j.$
\begin{align*}
W^Tr_j &= W^Tr_0-(W^TAW)(\hat{\xi}_j-\Delta_j\hat{\eta}_j)-W^T(V_jT_j+\sigma_{j+1}v_{j+1}e^T_j)\hat{\eta}_j \\
\implies 0 &= -(W^TAW)\hat{\xi}_j - (W^TAW)\Delta_j\hat{\eta}_j \\
\implies (&W^TAW)\hat{\xi}_j + (W^TAW)\Delta_j\hat{\eta}_j = 0 \implies \hat{\xi}_j = -\Delta_j\hat{\eta}_j
\end{align*}
Therefore, we have $$x_j = x_0-W\Delta_j\hat{\eta}_j+V_j\hat{\eta}_j$$ and $$r_j = r_0-(V_jT_j+\sigma_{j+1}v_{j+1}e^T_j)\hat{\eta}_j.$$
We know, $V_j^Tr_j=0\implies 0 = V_j^Tr_0-T_j\hat{\eta}_j\implies \hat{\eta}_j = \|r_0\|_2T^{-1}_je_1$. Now substituting this equation in the above equation, we get
\begin{align*}
r_j &= r_0-\|r_0\|_2V_jT_j(T^{-1}_je_1)-\|r_0\|_2\sigma_{j+1}v_{j+1}e^T_j(T^{-1}_je_1) \\
\implies r_j &= (-\|r_0\|_2\sigma_{j+1}e^T_jT^{-1}_je_1)v_{j+1} \\
\implies r_j &= c_jv_{j+1}
\end{align*}
\end{proof}

Let $T_j=L_jD_jL^T_j$ and define $P_j\equiv [p_0\;p_1\;\cdots\;p_{j-1}]=(-W\Delta_j+V_j)L^{-T}_j\Lambda_j$, where $\Lambda_j=\text{diag}\{c_0,c_1,\cdots,c_{j-1}\}.$

\begin{prop}
The approximate solution $x_j$, the residual $r_j$ and the descent direction $p_j$ satisfy the following recurrence relations:
\begin{align*}
x_j=x_{j-1}+\alpha_{j-1}p_{j-1},\;r_j=r_{j-1}-\alpha_{j-1}Ap_{j-1},\;p_j=r_j+\beta_{j-1}p_{j-1}-W\hat{\mu}_j,
\end{align*}
for some $\alpha_{j-1},\beta_{j-1},\hat{\mu}_j.$ Thus $\mathcal{K}_{k,j}(A,W,r_0)=\text{span}\{W,p_0,p_1,\cdots,p_{j-1}\}$.
\end{prop}

\begin{proof}
Let $\hat{\zeta}_j=\|r_0\|_2(L_jD_j\Lambda_j)^{-1}e_1$. We know,
\begin{align*}
x_j&=x_0+\|r_0\|_2(-W\Delta_j+V_j)T^{-1}_je_1\\
&=x_0+[(-W\Delta_j+V_j)L^{-T}_j\Lambda_j][\|r_0\|_2\Lambda^{-1}_jL^{T}_j(L_jD_jL^T_j)^{-1}_je_1] \\
&= x_0+P_j\zeta_j
\end{align*}
Now notice that $L_jD_j\Lambda_j$ is lower bidiagonal matrix, then we have,
$$\zeta_j=\left[ \begin{array}{c} \zeta_{j-1} \\ \alpha_{j-1} \end{array} \right],\text{ for some scalar }\alpha_{j-1}.$$
Hence, $$x_j = x_{j-1}+\alpha_{j-1}p_{j-1}\text{ and }r_j = r_{j-1}-\alpha_{j-1}Ap_{j-1}.$$
We know that $P_j=(-W\Delta_j+V_j)L^{-T}_j\Lambda_j,$ then 
$$P_j\Lambda^{-1}_jL^{T}_j\Lambda_j=(-W\Delta_j\Lambda_j+V_j\Lambda_j).$$
Now notice that $\Lambda^{-1}_jL^{T}_j\Lambda_j$ is a unit upper bidiagonal matrix. Comparing the last columns of the both sides of this equation 
$$p_{j-1}-\beta_{j-2}p_{j-2}=-W\hat{\mu}_{j-1}+c_{j-1}v_j,$$
where $\hat{\mu}_{j-1}=c_{j-1}\hat{v}_j\;and\;\beta_{j-2}=-c_{j-1}u_{j-1,j}/c_{j-2}.$
\end{proof}

\begin{prop}
The vectors $p_j$ are A-orthogonal to each other, i.e, $P^T_jAP_j$ is diagonal. In addition, they are also A-orthogonal to all $w_i$'s, i.e, $W^TAP_j=0.$
\end{prop}

\begin{proof}
\begin{align*}
P^T_jAP_j&= P^T_j(-AW\Delta_j+AV_j)L^{-T}_j\Lambda_j \\
&= P^T_j(V_jT_j+\sigma_{j+1}v_{j+1}e^T_j)L^{-T}_j\Lambda_j \\
&= \Lambda_jL^{-1}_j(-\Delta^T_jW^T+V^T_j)(V_jT_j+\sigma_{j+1}v_{j+1}e^T_j)L^{-T}_j\Lambda_j \\
&= \Lambda_jL^{-1}_jT_jL^{-T}_j\Lambda_j = \Lambda^2_jD_j\text{, which is diagonal.}
\end{align*}
Consider, 
$$W^TAP_j=W^T(V_jT_j+\sigma_{j+1}v_{j+1}e^T_j)L^{-T}_j\Lambda_j=0.$$
\end{proof}

\begin{prop}
The coefficients in Delfated-CG satisfy the relations
$$\alpha_j=\dfrac{(r_j,r_j)}{(Ap_j,p_j)},\quad \hat{\mu}_j=(W^TAW)^{-1}Ar_j,\quad \beta_j=\dfrac{(r_{j+1},r_{j+1})}{(r_j,r_j)}.$$
\end{prop}

\begin{proof}
From the orthogonality constraints on residuals, $r^T_{j-1}r_j=0.$ Then,
\begin{align*}
&\implies r^T_{j-1}(r_{j-1}-\alpha_{j-1}Ap_{j-1}) = 0 \\
&\implies \alpha_{j-1}=\dfrac{(r_{j-1},r_{j-1})}{(r_{j-1},Ap_{j-1})} \\
\text{But, }&(p_{j-2},Ap_{j-1})=0\;and\;(W\hat{\mu}_{j-1},Ap_{j-1})=0 \\
&\implies \alpha_{j-1}=\dfrac{(r_{j-1},r_{j-1})}{(\beta_{j-2}p_{j-2}+r_{j-1}-W\hat{\mu}_{j-1},Ap_{j-1})} \\
&\implies \alpha_{j-1}=\dfrac{(r_{j-1},r_{j-1})}{(p_{j-1},Ap_{j-1})}
\end{align*}
We know, 
\begin{align*}
p_j&=r_j+\beta_{j-1}p_{j-1}-W\hat{\mu}_j \\
\implies W^TAp_j&=W^TAr_j+\beta_{j-1}W^TAp_{j-1}-W^TAW\hat{\mu}_j \\
\implies \hat{\mu}_j&=(W^TAW)^{-1}Ar_j \\
\text{Similarly, }p^T_{j-1}Ap_j&=p^T_{j-1}Ar_j+\beta_{j-1}p^T_{j-1}Ap_{j-1}-p^T_{j-1}AW\hat{\mu}_j \\
\implies \beta_{j-1} &= -\dfrac{(Ap_{j-1},r_j)}{(p_{j-1},Ap_{j-1})}=\dfrac{1}{\alpha_{j-1}}\dfrac{(r_j-r_{j-1},r_j)}{(p_{j-1},Ap_{j-1})} \\
\implies \beta_{j-1} &= \dfrac{(r_{j+1},r_{j+1})}{(r_j,r_j)}
\end{align*}
\end{proof}

\begin{remark}
To guarantee that initial guess $x_0$ satisfies $W^Tr_0=0$, we can choose $x_0$ in the form 
$$x_0=x_{-1}+W(W^TAW)^{-1}W^Tr_{-1},$$
where $x_{-1}$ is arbitrary and $r_{-1}=b-Ax_{-1}$. \\
Then, $$W^Tr_0 = W^Tr_{-1}-(W^TAW)(W^TAW)^{-1}W^Tr_{-1}=0.$$
\end{remark}

Putting these relation together gives the following algorithm.

\newpage

\begin{algorithm}
\caption{Deflated-CG}
\begin{algorithmic}[1]
\State Choose k linearly independent vectors $w_1,w_2,\cdots,w_k$ and \\ \qquad define $W=[w_1\;w_2\;\cdots\;w_k]$
\State Choose an initial guess $x_0$ such that $W^Tr_0=0,$ where $r_0=b-Ax_0$
\State Solve $W^TAW\hat{\mu}_0=W^TAr_0$ for $\hat{\mu}_0$ and set $p_0=r_0-W\hat{\mu}_0$
\For{$j= 1,2,\cdots,m$,}
	\State $\alpha_{j-1}=(r_{j-1},r_{j-1})/(Ap_{j-1},p_{j-1})$
	\State $x_{j}=x_{j-1}+\alpha_{j-1}p_{j-1}$
	\State $r_{j}=r_{j-1}-\alpha_{j-1}Ap_{j-1}$
	\State $\beta_{j-1}=(r_j,r_j)/(r_{j-1},r_{j-1})$
	\State Solve $W^TAW\hat{\mu}_j=W^TAr_j$ for $\hat{\mu}_j$
	\State $p_{j}=r_{j}+\beta_{j-1}p_{j-1}-W\hat{\mu}_j$
\EndFor
\State EndDo
\end{algorithmic}
\end{algorithm}

\end{document}