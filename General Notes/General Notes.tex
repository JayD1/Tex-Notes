\documentclass[11pt,a4paper]{article}
\usepackage[margin=1in]{geometry}
\usepackage[utf8]{inputenc}
\usepackage{amsmath}
\usepackage[english]{babel}
\usepackage{amsthm}
\usepackage{amsfonts}
\usepackage{algorithm}
\usepackage{algpseudocode}
\usepackage{amssymb}
\usepackage{enumerate}
\usepackage[hidelinks]{hyperref}
\usepackage{cancel}

\hypersetup{
    colorlinks=true,
    linkcolor=blue,
    filecolor=magenta,      
    urlcolor=blue,
    pdftitle={General Notes},
    pdfpagemode=FullScreen,
}


\author{Jayadev Naram}
\title{General Notes} 

\begin{document}

\date{}
\maketitle
\tableofcontents
\newpage

\theoremstyle{plain}
\newtheorem{theorem}{Theorem}[section]
\newtheorem{corollary}{Corollary}[theorem]

\newtheorem{lemma}[theorem]{Lemma}
\newtheorem{proposition}[theorem]{Proposition}
\newtheorem{assume}{Assumption}

\theoremstyle{definition}
\newtheorem{definition}[theorem]{Definition}
\newtheorem{example}[theorem]{Example}
\newtheorem{remark}[theorem]{Remark}

\newcommand{\R}{\mathbb{R}}
\newcommand{\B}{\mathbb{B}}
\newcommand{\A}{\mathcal{A}}
\newcommand{\M}{\mathcal{M}}
\newcommand{\N}{\mathcal{N}}
\newcommand{\h}{\mathcal{H}}
\newcommand{\T}{\mathcal{T}}
\newcommand{\Pspace}{\mathbb{P}}
\newcommand{\Dist}{\mathcal{D}}
\newcommand{\perpProj}{\mathcal{P}^\perp}
\newcommand{\bb}{\mathbb{B}}
\newcommand{\Sprod}{\mathbb{S}_{xy}}
\newcommand{\highlight}[1]{\underline{\textit{\textbf{#1}}}}
\newcommand{\mapping}[3]{#1:#2\rightarrow #3}
\newcommand{\doubt}{\highlight{[??]}}
% \newcommand{\bigvert}[2]{#1{\raisebox{-.5ex}{$|$}_{#2}}}
\newcommand{\bigvert}[2]{\left.#1\right|_{#2}}
\newcommand{\sdnn}[1]{${#1}$}
\newcommand{\bsdnn}[1]{$\boldsymbol{#1}$}
\newcommand{\ifthen}[2]{\textbf{(#1)}\boldsymbol{\implies}\textbf{(#2)}}
\newcommand{\bsdn}[1]{\boldsymbol{#1}}
\newcommand{\forward}{$(\implies)$\ }
\newcommand{\converse}{$(\impliedby)$\ }
\newcommand{\Lt}[1]{\underset{#1\rightarrow 0}{Lt}}
\newcommand{\norm}[1]{\|#1\|}
\newcommand{\dparder}[2]{\dfrac{\partial #1}{\partial x_{#2}}}
\newcommand{\fparder}[2]{\frac{\partial #1}{\partial x_{#2}}}
\newcommand{\parder}[2]{\partial #1/\partial x_{#2}}
\newcommand{\parop}[1]{\dfrac{\partial}{\partial x_{#1}}}
\newcommand{\innerproduct}[2]{\langle #1, #2 \rangle}
\newcommand{\metric}[2]{[#1, #2]}
\newcommand{\genst}{St_B(n,p)}
\newcommand{\igenst}[1]{St_{B_{#1}}(n_{#1},p)}
\newcommand{\realmat}[2]{\R^{#1\times #2}}
\newcommand{\Skew}{\mathcal{S}_{skew}(p)}
\newcommand{\Sym}{\mathcal{S}_{sym}(p)}
\newcommand{\XperpB}{X_{B^\perp}}
\newcommand{\polarRetr}{R^{polar}_X}
\newcommand{\qrRetr}{R^{QR}_X}
\newcommand{\vectransport}{\mathcal{T}}
\newcommand{\grad}{\mathrm{grad}\,}
\newcommand{\hess}{\mathrm{Hess}\,}
\newcommand{\trace}{\mathrm{tr}}

\section{Inequalities for the Wasserstein Mean of Positive Definite Matrices}

\begin{definition}
    Let $\Pspace$ be the space of $n\times n$ complex positive definite matrices. The \highlight{Bures-} \highlight{Wasserstein distance} on $\Pspace$ is the metric defined as 
    \begin{equation}\label{eqn:bw_dist}
        d(A,B) = \Big[\trace(A+B) - 2\trace(A^{1/2}BA^{1/2})^{1/2} \Big]^{1/2}.
    \end{equation}
\end{definition}

\begin{definition}
    Let $A_1,\ldots, A_m$ be given positive definite matrices and let $w = (w_1,\ldots, w_m)$ be a vector of weights, i.e., $w_j\ge 0$ and $\sum_{j=1}^m w_j = 1$. Then the \highlight{(weighted) Wasserstein mean}, or the \highlight{Wasserstein barycentre} of $A_1,\ldots,A_m$ is defined as 
    \begin{equation}\label{eqn:wasserstein_barycentre}
        \Omega(w; A_1,\ldots,A_m) = \underset{X\in\Pspace}{\arg\min}\sum_{j=1}^m w_j d^2(X,A_j).
    \end{equation}
\end{definition}

\begin{remark}
    The function defined on $\Pspace$ by the sum on the right hand side of \eqref{eqn:wasserstein_barycentre} has a unique minimizer. 
\end{remark}

\begin{remark}
    The Wasserstein barycentre $\Omega$ defined in \eqref{eqn:wasserstein_barycentre} is the unique positive definite solution of the equation 
    \begin{equation}
        X = \sum_{j=1}^m w_j(X^{1/2}A_jX^{1/2})^{1/2}.
    \end{equation}
    Consider the special case $m = 2$, writing $(A_1,A_2) = (A,B)$ and $(w_1,w_2) = (1-t, t)$, where $0\le t\le 1$. We rewrite the above equation as
    \begin{align*}
        &\qquad\ \ \, X = (1-t) (X^{1/2}AX^{1/2})^{1/2} + t (X^{1/2}BX^{1/2})^{1/2} \\
        &\implies X^{1/4} X^{1/2} X^{1/4} = X^{1/4} ((1-t)A^{1/2}+tB^{1/2}) X^{1/4}\\
        &\implies X^{1/2} = (1-t)A^{1/2}+tB^{1/2}\\
        &\implies X = ((1-t)A^{1/2}+tB^{1/2})^2.
    \end{align*}
    Thus, we obtained an explicit formula for $\Omega$. Denoting this by $A\diamondsuit_t B$ we have
    \begin{equation}\label{eqn:bw_geodesic}
        A\diamondsuit_t B = (1-t)^2 A + t^2 B + t(1-t)[(AB)^{1/2}+(BA)^{1/2}].
    \end{equation}
    The metric $d$ in \eqref{eqn:bw_dist} is the distance corresponding to an underlying metric on $\Pspace$, and \eqref{eqn:bw_geodesic} is an equation for the geodesic segment between two points $A$ and $B$ in the manifold $\Pspace$. The special choice $t=1/2$ gives the midpoint of this geodesic. This is denoted by 
    \begin{equation}\label{eqn:wasserstein_mean_midpoint}
        A\diamondsuit B = \dfrac{1}{4} [A+B+(AB)^{1/2}+(BA)^{1/2}],
    \end{equation}
    and can be thought of as the \highlight{Wasserstein mean} of $A$ and $B$.
\end{remark}

\begin{definition}
    The \highlight{Cartan metric} 
    \begin{equation}\label{eqn:cartan_metric}
        \delta(A,B) = \|\log A^{-1/2}BA^{-1/2}\|_2,
    \end{equation}
    where $\|\cdot\|_2$ is the Forbenius norm on matrices. The \highlight{weighted Cartan mean} (or the \highlight{weighted} \highlight{geometric mean}) of $A_1,\ldots, A_m$ is defined as
    \begin{equation}\label{eqn:cartan_mean}
        G(w;A_1,\ldots,A_m) = \underset{X\in\Pspace}{\arg\min} \sum_{j=1}^m w_k\delta^2(X,A_j).
    \end{equation}
\end{definition}

\begin{remark}
    The (unique) solution of this minimization problem is also the positive definite solution of the equation
    \begin{equation}
        \sum_{j=1}^m w_j\log (X^{-1/2}A_j X^{-1/2}) = 0.
    \end{equation}
    Analogous to \eqref{eqn:bw_geodesic} the equation of the geodesic segment joining $A$ and $B$ with respect to the metric $\delta$ is 
    \begin{equation}\label{eqn:cartain_geodesic}
        A\#_t B = A^{1/2}(A^{-1/2}B A^{-1/2})^t A^{1/2}, \qquad 0\le t\le 1.
    \end{equation}
    This is also the $t$-weighted geometric mean of $A$ and $B$. When $t = 1/2$, this reduces to 
    \begin{equation}\label{eqn:cartain_mean_midpoint}
        A\# B = A^{1/2}(A^{-1/2}B A^{-1/2})^{1/2} A^{1/2},
    \end{equation}
    and is called the geometric mean of $A$ and $B$.
\end{remark}

% \newpage

% \begin{thebibliography}{9}
% \bibitem{Rockafellar} 
%     Rockafellar, R.T. Theory of Optimization. Princeton University Press, 1972.
    
% \bibitem{Fenchel}
%     Fenchel, W. Convex Cones, Sets, and Functions. Princeton University, 1953.
    
    
% \end{thebibliography}
    

\end{document}